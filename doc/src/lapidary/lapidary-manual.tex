\device{postscript}
\make{manual}
\disable{figurecontents}
\libraryfile{Stable}
\libraryfile{Garnet}
\string{TitleString = `Lapidary'}
\use{Bibliography = `garnet.bib'}

\begin{titlepage}
\begin{titlebox}
\vspace{0.6 inch}
\bg{Lapidary Reference Manual}

{\bf Brad T. Vander Zanden}
{\bf David Bolt}
\vspace{0.3 line}
\value{date}
\end{titlebox}
\vspace{0.5 inch}
\begin{center}
{\bf Abstract}\end{center}
\begin{text}
This document describes the features and operations provided by Lapidary,
a graphical interface builder that allows a user to pictorially specify
all graphical aspects of an application and interactively create much
of the behavior. Lapidary allows a user to draw most of Opal's objects,
combine them into aggregadgets, align them using iconic constraint
menus or custom constraints, and create behaviors by entering appropriate
parameters in dialog boxes representing each of Garnet's interactors, or
by demonstrating the appropriate behavior for feedback objects.

\vspace{0.3 inch}
\begin{center}
Copyright \copyright{1994} - Carnegie Mellon University
\end{center}

%\vspace{2 lines}
The original Garnet research
was sponsored by NCCOSC under Contract No.
N66001-94-C-6037, Arpa Order No. B326.
The views and conclusions contained in this document are those of
the authors and should not be interpreted as representing the
official policies, either expressed or implied, of NCCOSC or the U.S.
Government.

Modifications Copyright \copyright{2005} by Robert P. Goldman and other
contributors.  All modifications to this document are made freely
available under the terms of the Lesser Gnu Public License.


\end{text}
\end{titlepage}


\string{overview = `1'} % \comment{26 pages}
\string{overview-first-page = `3'}
\string{apps = `27'} % \comment{12 pages}
\string{apps-first-page = `29'}
\string{tour = `41'} % \comment{20 pages}
\string{tour-first-page = `43'}
\string{tutorial = `61'} % \comment{42 pages}
\string{tutorial-first-page = `63'}
\string{kr = `101'} % \comment{52 pages}
\string{kr-first-page = `103'}
\string{opal = `151'} % \comment{70 pages}
\string{opal-first-page = `153'}
\string{inter = `219'} % \comment{78 pages}
\string{inter-first-page = `221'}
\string{aggregadgets = `295'} % \comment{54 pages}
\string{aggregadgets-first-page = `297'}
\string{gadgets = `347'} % \comment{116 pages}
\string{gadgets-first-page = `349'}
\string{debug = `461'} % \comment{20 pages}
\string{debug-first-page = `463'}
\string{demos = `481'} % \comment{10 pages}
\string{demos-first-page = `483'}
\string{sampleprog = `491'} % \comment{14 pages}
\string{sampleprog-first-page = `493'}
\string{gilt = `505'} % \comment{20 pages}
\string{gilt-first-page = `507'}
\string{c32 = `525'} % \comment{12 pages}
\string{c32-first-page = `527'}
\string{lapidary = `537'} % \comment{36 pages}
\string{lapidary-first-page = `539'}
\string{hints = `573'} % \comment{8 pages}
\string{hints-first-page = `575'}
\string{GlobalIndex = `580'}

\set{page = lapidary-first-page}

\chapter{Getting Started}

\index{Run Lapidary}
\index{Start Lapidary}
To load Lapidary, type {\tt (load garnet-lapidary-loader}) after Garnet has
been loaded, or type {\tt (defvar load-lapidary-p t}) before Garnet
is loaded, and Garnet will automatically load Lapidary when the Garnet loader
file is invoked.  To start Lapidary, type {\tt (lapidary:do-go}).
This will cause Lapidary to come up in its initial state
with the following windows:

\begin{itemize}
\item editor-menu: This menu contains a set of functions that deal with aggregadgets,
constraints, saving and restoring objects, deleting objects,
and setting properties of objects.

\item shapes menu: This menu allows the designer to create opal graphical objects
and windows.

\item box-constraint menu: This menu allows the designer to attach constraints to
an object that control its left, top, width, and height.

\item drawing window: This window allows the designer to create new
objects or load objects from existing files.

\end{itemize}

\section{Object Creation}
Lapidary allows new objects to be created from scratch,
loaded from pre-defined gadgets files, or created directly in Garnet
and then linked to a Lapidary window.  The shapes menu displays the
primitive graphical objects that can be created in Lapidary.

\begin{figure}
\begin{center}
\graphic{postscript = `lapidary/shapes.PS', Magnify = 0.5, BoundingBox = File}\end{center}
\caption{Shapes menu}
\end{figure}

\index{line}
\index{rectangle}
\index{roundtangle}
\index{circle}
\index{text}
\index{text}
\index{multi-text}
\index{window}
\index{bitmap}
\index{aggrelist}
The first six geometric shapes can be created by selecting the
appropriate menu-item
and sweeping out the item in a drawing window with the right mouse button down.
Feedback corresponding to the selected shape will be shown as the
object is swept out.   Properties  such as line-style, filling-style, and
draw-function can be set from the corresponding property menus (see
section \ref{Properties}).

To create a single line of text, select text and then click where
you want the text to start.  A cursor will
appear and one line of text can be entered from the keyboard.  For more
than one line of text use multi-text.  Single-line text can be terminated
with either a mouse click or by hitting RETURN but, multi-line text
can only be terminated by a mouse click.

To create a window, select the {\tt window} menu-item and Lapidary will
create a new window.  Since, new windows initially have the same
size and location as the draw window,
they must be moved in order to expose the original draw window.

Bitmaps can be loaded by selecting the {\tt bitmap} menu-item. Lapidary
brings up a dialog box that allows the user to enter the name of an image file
and the window that the bitmap should be placed in.  The window name
is obtained from the title border that surrounds a window or the name
that appears in the icon for the window.

To create a horizontal or vertical list, first select a prototype
object.  Then select horizontal or vertical list and sweep out the list.
A property sheet will appear that can be used to set parameters that
control the list's appearance.  A description of the parameters can
be found in the chapter on aggregadgets and aggrelists.


\section{Selecting Objects}
\index{selecting objects}
\index{primary selection}
\index{secondary selection}
Lapidary permits two types of selections: primary selections and secondary
selections.
Primary selections are denoted by black grow boxes that sprout around the
perimeter of an object; secondary selections are denoted by white grow
boxes.  Most operations do not distinguish between these two types of
selections and will operate in the same way on both types of selections.
However, two operations, attaching a constraint to an object and
defining parameters for an object, do make this distinction.

Lapidary provides two types of selection modes (Figure
\ref{lapidary-editor-menu}): a ``leaves'' mode
which causes Lapidary to select leaf
elements of an aggregate, and a ``top-level objects'' mode which causes
Lapidary to select top-level aggregates (objects that do not belong to an
aggregate will be selected in either mode).
\index{aggregadget, selection}
To aid the user in determining whether they have selected a leaf or
aggregate element, Lapidary uses different types of selection
handles\dashrectangular handles for leaf elements
and circular
handles for aggregates (Figure \ref{selection}).
If the object is too small to
accomodate 8 selection handles, either a thin or thick-lined arrow is used
to highlight the selection,
depending on whether the object is
a leaf or aggregate object (Figure \ref{selection}).

In either mode, additional
clicks over the selected object will cause Lapidary to cycle through
the aggregate hierarchy.
For example, when the
user clicks on the label shown in Figure \ref{aggregate-hierarchy}.a,
and Lapidary is in ``top-level
objects'' mode, the entire list element is selected
(Figure \ref{selection-techniques}.a). If the user
clicks on the label again, the label is selected
(Figure \ref{selection-techniques}.b). Clicking once
more with the mouse causes the key-box to become selected (Figure
\ref{selection-techniques}.c). Finally, one more click causes the
list element to be selected, at
which point the cycle repeats itself. In ``leaves'' mode, the label would be
the first object selected, then the key box, and finally the list element.

\begin{figure}

\bar{}
% \comment{l, bot, l->r,b->t}
\begin{center}
\graphic{Postscript=`lapidary/editor-win.PS',boundingbox=file,
                    magnify=0.7}\end{center}
\caption{
The user can cause Lapidary to select leaves of aggregates or top-level
aggregates by choosing the appropriate selection mode in the editor
menu window.
}
\tag{lapidary-editor-menu}
\bar{}

\end{figure}


\begin{figure}
% \comment{l, bot, l->r,b->t}
\begin{center}
\graphic{Postscript=`lapidary/agg-hierarchy-a.PS',Magnify=0.7,
	boundingbox=File}\graphic{
	   Postscript=`lapidary/agg-hierarchy-b.PS',Magnify=0.7,
	   boundingbox=File}\end{center}
\begin{tabular}
\tabset{0.65in, 3.5in}\\
 & (a) & (b)\\
\Tabclear\\
\end{tabular}
\caption{
A list element (a) and the objects used to build this list element (b).
}
\tag{aggregate-hierarchy}
\bar{}
\end{figure}

\begin{figure}
\bar{}
\begin{center}
\graphic{Postscript=`lapidary/selection.PS',magnify=0.6,
        boundingbox=File}\end{center}
\begin{tabular}
\tabset{1.875in, 3in, 4.375in}\\
 & (a) & (b) & (c)\\
\Tabclear\\
\end{tabular}
\caption{
As the user repeatedly clicks the mouse button over an object, the selection
cycles through the aggregate hierarchy shown in Figure
\ref{aggregate-hierarchy}. If Lapidary is in ``top-level objects'' mode,
then the list element is initially
selected (a). A second click selects the label (b), and a third click
selects the key-box (c).
}
\bar{}
\tag{selection-techniques}
\end{figure}

Lapidary provides the usual range of selection
operations found in drawing editors\dashselect,
add to selection, deselect, remove from selection,
select/deselect in region. These operations are supported for
both primary and secondary selections.
In addition, Lapidary allows the user to select covered objects by pointing
at an already selected object and requesting that the object directly covered
by the selected object be selected.

Section \ref{mouse-commands} provides specific details on each of the
selection operations.

\begin{figure}
\begin{center}
\graphic{postscript = `lapidary/selections.PS', BoundingBox = File}\end{center}
\caption{Different types of selection objects.  Squares are for primitive
graphical objects, circles are for aggregadgets, and arrows are for objects
too small to accommodate grow boxes.}
\tag{selection}
\end{figure}

\section{Mouse-Based Commands}
\index{mouse}
\label{mouse-commands}
Lapidary is primarily a mouse-based system so it is important to know which
mouse buttons correspond to which operation.
These bindings are set in the file mouse-bindings.lisp and may be
edited.  Currently the following operations can
be bound to mouse buttons (the pair following each entry shows the
default and the variable that must be changed to modify the default):

\begin{itemize}
\item \index{primary selection}
Primary Selection (leftdown, *prim-select-one-obj*): The user can
either point at a particular object
and make it the primary selection, or sweep out a rectangular region of the
screen and make all objects that {\it intersect} the
region be primary selections.
This operation causes the previous primary selections to be deselected. If the
mouse is not pointing at any objects, all primary selections are deselected.
Each successive mouse click over the same object
moves the selection one level higher in
the aggregate
hierarchy, until the top-most level is reached, at which point the
selection process cycles back to a leaf (if the selection is initially a
top-level object, the next click cycles to the leaf).

\item \index{Secondary Selection}
Secondary Selection (middledown, *sec-select-one-obj*): Same as primary
selection but makes a secondary selection.

\item \index{primary selection, deselect}
Deselect Primary Selection (control-leftdown, *primary-deselection-button*):
This operation allows the user to deselect primary selections. The user
can either point at a specific object or sweep out a rectangular region,
in which case all objects that intersect this region will be
deselected (if they are primary selections).

\item \index{secondary selection, deselect}
Deselect Secondary Selection
(control-middledown, *secondary-deselection-button*): Same as Deselect
Primary Selection except secondary selections are deselected.

\item \index{primary select covered object}
Primary Select Covered Object (shift-control-leftdown,
*prim-push-sel-under-button*): This operation
allows the user to select covered objects. When the user points at a particular
area of the screen, Lapidary determines which object is currently selected,
and then deselects it and primary selects the first object that it covers. If
no object under the mouse is selected, Lapidary primary selects the top object.
If multiple objects under the mouse are selected, Lapidary finds the first
unselected object which is under a selected object, selects the unselected
object, and deselects the topmost selected object.

\item \index{secondary select covered object}
Secondary Select Covered Object (shift-control-middledown,
*sec-push-sel-under-button*): Same as Primary Select Covered Object except
a secondary selection is made.

\item \index{Primary Selection, add to}
Add to Primary Selection (shift-leftdown, *prim-add-to-select*): Same as
Primary Selection except previously selected objects remain selected.
Covered objects that are selected are
automatically added to a selection, rather than causing
previously selected objects to be deselected. Multiple clicks with
the {\tt Add to Primary Selection} button over the {\it selection handles}
of a covered object will cause the selection to cycle through the
aggregate hierarchy (Figure \ref{covered-selection}).

\item \index{secondary selection, add to}
Add to Secondary Selection (shift-middledown, *sec-add-to-select*): Same
as add to primary selection but adds to secondary selection.

\item \index{Move}
Move Object (leftdown, *move-button*): This operation allows
the user to move an
object around the window. The user must point at one of the eight ``grow''
boxes around the perimeter of box objects, or one of the three ``grow''
boxes attached to line objects or the arrow if the object is too
small to contain grow boxes. If the object is a box object
and the user points at one of
the corner boxes, the object can move in any direction, if the user points
at one of the side boxes, the object can move in only one direction (along
the x-axis if the left or right side is chosen and along the y-axis if the
top or bottom side is chosen). If the object is a line object, Lapidary will
attach the mouse cursor to the point designated by the grow box (either an
endpoint of the line or its midpoint) and move the line in any direction.
If the object is undersized so that the object does not have grow boxes but
instead is pointed at by an arrow, then pointing at the arrow will cause the
cursor to be attached to the northwest corner of the object and the object
can be moved in any direction.

\item \index{grow}
\index{resize}
Grow Object (middledown, *grow-button*): This operation allows
the user to resize an
object. The user must point at one of the eight ``grow''
boxes around the perimeter of the object if the object is a box, one of
the endpoint ``grow'' boxes attached to the object if the object is a line,
or the arrow that points at the object if the object is too small to contain
the grow boxes. If the object is a box object and
the user points at one of
the corner boxes, both the object's width and height can change, if the
user points
at one of the side boxes, only one of the object's dimensions will change
(the width if the left or right side is chosen, the height if the top or
bottom side is chosen). If the object is a line object, Lapidary will
attach the mouse cursor to the point designated by the grow box and move
that endpoint while holding the other endpoint fixed.
If the object is undersized so that the object does not have grow boxes but
instead is pointed at by an arrow, then pointing at the arrow will cause the
cursor to be attached to the northwest corner of the object and the object's
width and height will both change.

\item \index{create object}
Object Creation (rightdown, *obj-creation-button*): The user sweeps out
a region of the screen and Lapidary creates the object selected
in the shapes menu.

\item \index{make copy}
Copy Object (shift-rightdown, *copy-button*):
This operation allows
the user to create a copy of an object and position it in
a window (copies of an object can also be created using the
{\tt make copy} command in the editor menu window\dashsee
Section \ref{edit-commands} for details).
The user must point at one of the eight ``grow''
boxes around the perimeter of box objects, or one of the three ``grow''
boxes attached to line objects or the arrow if the object is too
small to contain grow boxes.
The selected ``grow'' box
constrains the initial movement of the new object (see {\tt Move Object} for
a description of how the ``grow'' boxes constrain movement).

\item \index{make instance}
Instance Object (control-rightdown, *instance-button*):
This operation allows
the user to create an instance of an object and position it in
a window (instances of an object can also be created using the
{\tt make instance} command in the editor menu window\dashsee
Section \ref{edit-commands} for details).
The user must point at one of the eight ``grow''
boxes around the perimeter of box objects, or one of the three ``grow''
boxes attached to line objects or the arrow if the object is too
small to contain grow boxes.
The selected ``grow'' box
constrains the initial movement of the new object (see {\tt Move Object} for
a description of how the ``grow'' boxes constrain movement).

\item \index{text, edit}
Text Editing (rightdown, *obj-creation-button*): The user can edit
a selected text object by pointing at it and clicking with the object
creation button.
The user can use any text editing command described in the interactors
manual and clicks down on the mouse button to indicate that editing
is complete.

\end{itemize}

\begin{figure}
\bar{}
\begin{center}
\graphic{Postscript=`lapidary/covered-selection.PS',magnify=0.6,
        boundingbox=File}\end{center}
\begin{tabular}
\tabset{1.8175in, 3.125in, 4.25in}\\
 & (a) & (b) & (c)\\
\Tabclear\\
\end{tabular}
\caption{Selecting a covered object in ``leaves'' mode.
The label is covered by an xor feedback object, so the feedback object is
the initial selection (a). Clicking the {\tt shift-control-leftdown} mouse button
pushes the selection down to the covered label (b). Clicking the
add to selection button ({\tt shift-leftdown} over the feedback arrow
causes the selection to cycle up to the next level in the aggregate
hierarchy, in this case, the key-box (c).
}
\bar{}
\tag{covered-selection}
\end{figure}

\begin{figure}
\bar{}
\begin{center}
\graphic{Postscript=`lapidary/duplicating-a.PS',magnify=0.35,
        boundingbox=File}\graphic{
        Postscript=`lapidary/duplicating-b.PS',magnify=0.35,
        boundingbox=File}\graphic{
        Postscript=`lapidary/duplicating-c.PS',magnify=0.35,
        boundingbox=File}\end{center}
\begin{tabular}
\tabset{1.125in, 3.25in, 5.25in}\\
 & (a) & (b) & (c)\\
\Tabclear\\
\end{tabular}
\caption{
Objects can be duplicated or instanced by clicking on one of the
selection handles (a), dragging the new object to the appropriate
location (b), and dropping it (c).
}
\bar{}
\tag{duplicating}
\end{figure}



\chapter{Editor Menu Commands}
\label{editor-menu}

The commands in Lapidary's pull down menu (Figure \ref{lapidary-editor-menu})
provide a set of commands for saving and restoring objects, manipulating
aggregadgets, applying constraints, and editing properties.

\section{File}
\index{save}
\begin{itemize}
\item 
{\tt Save Gadget:} Objects are written out using {\tt opal:write-gadget},
so the file contains
a series of create-instance calls. The value in the object's {\tt :known-as}
slot is passed as the name parameter to create-instance. For example, if
the object's {\tt :known-as} slot is {\tt :white-rect} and the object is a
rectangle, the first line of the create-instance would be

\begin{programexample}
(create-instance 'white-rect opal:rectangle)
\end{programexample}

Primary selections are saved before secondary selections, so it is best to
make prototypes primary selections and instances of these prototypes secondary
selections.  The user can also save an entire window by having no
objects selected and typing in the string that appears in a window's title
bar or icon in the corresponding area of the dialog box.

Lapidary looks at each saved object to determine if the object has any
links which Lapidary thinks should be parameters. If Lapidary finds any
such links, it pops up the link parameters dialog box and asks the user
if these links should be made into parameters (see Section \ref{parameters}).
Pressing either the {\tt OK} or {\tt CANCEL} buttons in the link parameters
dialog box allows Lapidary to continue. The {\tt CANCEL} button in the
link parameters dialog box will not cause Lapidary to discontinue the save
operation, it will simply cause Lapidary to proceed to the next object.

\begin{figure}
\begin{center}
\graphic{postscript = `lapidary/saving.PS', Magnify = 0.7,
                                BoundingBox = File}\end{center}
\caption{Save file dialog box}
\tag{save-dialog-box}
\end{figure}


\item 
{\tt Load Gadget:} Requests the name of a file and then loads it
(Figure \ref{load-dialog-box}).
Lapidary expects a variable named *Garnet-Objects-Just-Created* to
be initialized in the user package which contains the names of
the created objects. If the user selects the option {\tt Replace existing
objects}, then the objects in the loaded file will replace the current
objects in the drawing window. If the user selects the option
{\tt Add to existing objects}, then the objects in the file will be
added to the existing objects in the window.

\begin{figure}
\begin{center}
\graphic{postscript = `lapidary/loading.PS', Magnify = 0.5, BoundingBox
= File}\end{center}
\caption{Load file dialog box}
\tag{load-dialog-box}
\end{figure}


\item \index{add gadget}

{\tt Add Gadget:} Users may create objects in the lisp listener and then link
them to a Lapidary window.  {\tt add-gadget} pops up a
dialog box that requests the name of the object to be added
and the name of a window to place the object in (Figure \ref{add-gadget-fig}).
The name of
the object should be the one used in the call to create-instance.
For example, the object created by
{\tt (create-instance 'my-gadget opal:rectangle)} is named ``my-gadget''.
The name of the window should be the name that appears in the
window's title bar or in its icon.

\begin{figure}
\begin{center}
\graphic{postscript = `lapidary/add-gadget.PS', Magnify = 0.7, BoundingBox = File}\end{center}
\caption{Add gadget dialog box}
\tag{add-gadget-fig}
\end{figure}

The user has the option of either adding the object itself or an
instance of the object to Lapidary. If the user decides to add the
object itself and the object has instances, Lapidary will pop up
a warning box indicating that editing this object could have unintended
consequences on other applications that use this object. For example,
it is better to add an instance of a garnet gadgets text button rather
than the actual button defined in the gadgets package, since editing the
actual button is likely to cause Lapidary to fail
(Lapidary uses garnet gadgets text buttons).


\item \index{quit}
\index{exit}
\index{stop}
{\tt Quit:} Allows the user to exit Lapidary. It is suggested that before
rebooting Lapidary, that the user create a new lisp listener and reload
Garnet.

\end{itemize}


\section{Edit}
\index{Edit}
\label{edit-commands}
\begin{itemize}
\item \index{make instance}
{\tt Make Instance:} Creates an instance of the selected object.  The selected
object is the new object's prototype.

\item \index{make copy}
{\tt Make Copy:} Creates a copy of the selected object.  The value of each
slot in the selected object will be copied to the new-object.
The new object will have the same prototype as the selected object,
and thus will inherit from the selected object's prototype rather
than the selected object.

\item \index{delete object}
{\tt Delete Object:} Destroys all selected objects.

\item \index{delete window}
{\tt Delete Window:} Pops up a dialog box and asks the user to input
the name of a window that appears in a window's title bar or icon.
Lapidary then destroys the window.

\end{itemize}


\section{Properties}
\index{Properties}
\label{Properties}

Lapidary contains four property menus that control an object's line-style,
filling-style, draw-function, and font.
The line-style and filling-style menus (Figures \ref{shade-menu} and
\ref{line-menu}) provide a set of commonly used
styles, an ``Other'' option which prompts the user for the name of a style,
and a ``Constraint'' option that allows the user to enter a custom constraint
that defines the style (see Section \ref{constraints} for information
on how to enter a custom constraint).
The color button pops up a color menu that allows the user to select a
pre-defined color or create a new color by mixing hues of red, green,
and blue.

\begin{itemize}
\item 
\index{filling style}
{\tt Filling Style:} Allows the user to set the filling style of selected
objects.

\begin{figure}
\begin{center}
\graphic{postscript = `lapidary/filling-properties.PS', BoundingBox = File, magnify = 0.7}\end{center}
\caption{Filling styles that can be attached to objects in Lapidary}
\tag{shade-menu}
\end{figure}


\item \index{line style}

{\tt Line Style:} Allows the user to set the line style of selected objects.

\begin{figure}
\begin{center}
\graphic{postscript = `lapidary/line-properties.PS',  BoundingBox = File, magnify = 0.7}\end{center}
\caption{Line styles that can be attached to objects in Lapidary}
\tag{line-menu}
\end{figure}


\item \index{draw function}

{\tt Draw Function:} Allows the user to set the draw
function of all selected objects.  The Opal chapter describes draw
functions in more detail.

\begin{figure}
\begin{center}
\graphic{postscript = `lapidary/draw-function.PS', BoundingBox = File, magnify = 0.7} \end{center}
\caption{Draw functions that can be attached to objects in Lapidary}
\tag{draw-menu}
\end{figure}


\item \index{name object}
{\tt Name Object:} Requests a name from the user (no quotes should
be used), converts it to a keyword, and stores it in the :known-as slot
of the selected object (if there is more than one selected object, Lapidary
will rename the last object the user selected; name object does
not distinguish between primary and secondary selections).
Lapidary also creates a link with this name in the object's parent
that points to this object.  When an object is saved, it will be assigned
this name.

\item \index{list properties}
{\tt List Properties:} Brings up a property list for horizontal and vertical
lists. This property list allows the user to modify any of the customizable
slots of an aggrelist. The list of customizable slots can be found in
the Aggrelists chapter.

\item \index{text properties}

{\tt Text Properties:} Allows the user to choose a standard Opal font,
to request a font from one of the directories on the user's font path,
to request a font from an arbitrary directory, or to enter a custom
constraint that determines the font (Figure \ref{text-menu}).
It also allows the user to enter a custom constraint that determines the
string of a text object.

\begin{figure}
\begin{center}
\graphic{postscript = `lapidary/text-properties.PS', magnify = 0.7,
                                 BoundingBox = File}\end{center}
\caption{Lapidary's text properties menu}
\tag{text-menu}
\end{figure}


\item \index{parameters}

\label{parameters}
{\tt Parameters:} Allows the user to specify that
one or more slots in an object should be parameters
(Figure \ref{parameters-fig}). A slot that is a parameter will have
its value provided at run-time by the application.
To create parameters, the user must make both a primary and a secondary
selection. The primary selection is the object whose slots are being
made into parameters and the secondary selection is the object that
the parameters will retrieve their values from. Typically the secondary
selection will be the top-level aggregadget that contains the object,
since the top-level aggregadget is the only object that the application
should know about (an application should not be required to know the parts
of an aggregadget). For example, if a
label text object belongs to an aggregadget, the user might make the label
the primary selection and the aggregadget the secondary selection.
If the object is already at the top-level, then the
object should be both the primary and secondary selection.

\begin{figure}
\begin{center}
\graphic{postscript = `lapidary/parameters.PS', Magnify = 0.8,
                                                      BoundingBox = File}\end{center}
\caption{Parameters dialog box}
\tag{parameters-fig}
\end{figure}

To turn a slot into a parameter, select the text box next to the slot
and enter the name of the slot in the
secondary selection that the slot should retrieve its value from.
In Figure \ref{parameters-fig}, the label's {\tt string} slot
retrieves its value from list element's {\tt value} slot,
and the {\tt font} slot retrieves its value from the list element's
{\tt font} slot. To
make the slot no longer be a parameter, make the slot's text box be blank.
Lapidary maintains a list of slots for each objects that can be turned
into parameters. If the user wants to parameterize a slot that is not
displayed in the parameters dialog box, the user can bring up C32 and
place a formula in the desired slot that retrieves its value from the
top-level aggregadget.

\begin{figure}
\begin{center}
\graphic{postscript=`lapidary/link-parameters.PS',Magnify=0.8,BoundingBox=File}\end{center}
\caption{Link parameters dialog box}
\tag{link-parameters}
\end{figure}

The link parameters button in the parameters dialog box allows the user
to specify links that should be parameters.
Links are used by Lapidary-generated
constraints to indirectly reference other objects. For example, when
the user creates a constraint that attaches the endpoint of a line,
say {\tt arrow1} to a rectangle, say {\tt rect1},
Lapidary generates a link in
{\tt arrow1} that points to {\tt rect1}. When a link references an object that
is not part of the primary selection's top-level aggregadget, Lapidary
guesses that this link should be a parameter and displays it in the
link parameters dialog box (Figure \ref{link-parameters}). For each such
link, Lapidary displays the value of the link, the slots that reference
the link, and a parameter name, if any, that the user has assigned to this
link. The user can change this parameter name by editing it, or can
indicate that this link should not be a parameter by making the parameter
name blank.


\begin{figure}
\begin{center}
\graphic{postscript = `lapidary/ParamItems.PS', Magnify = 0.8,
                                                      BoundingBox = File}\end{center}
\caption{Parameters dialog box for an Aggrelist}
\tag{paramItems-fig}
\end{figure}

To make a slot depend on an :items list in an aggrelist, make any
object in the aggrelist be a primary selection and make the
aggrelist be the secondary selection. Then enter :items in
the labeled box for any slot on the parameters menu that should get
its value from :items.  For example, suppose the
prototype object for a list is a text object and
the string and font slots of the text object should retrieve their
values from the aggrelist's {\tt :items} slot. To do this the user
makes the aggrelist the secondary selection and one
of the text objects in the aggrelist a primary selection. The user
then selects the {\tt parameters} option, which
causes Lapidary to pop up the parameters menu. Typing :items in the
type-in fields next to the {\tt font} and {\tt string} slots creates
the necessary formulas that link these slots to the {\tt :items} slot in
the aggrelist (Figure \ref{paramItems-fig}). The :items slot of
the aggrelist will now contain a list of the form
((string1 font1) (string2 font2)...(stringN fontN)).

If the prototype object is an aggregadget (such as a labeled box that
contains a rectangle and a piece of text), then any of the parts of the
aggregadget, and the aggregadget itself, can have slots that depend on
the aggrelist's :items slot. This is done by parameterizing the parts
one at a time. For example, if the string slot of the text object and
the filling-style slot of the rectangle should be parameters, the user
could first select the rectangle and parameterize it, then select the
text object and parameterize it. Lapidary does not follow any easily
described rules in constructing the :items list (e.g., the string and
font values could easily have been reversed in the above list), so users
should look at the :items list Lapidary constructs before writing their own.

If a slotname besides
:items (e.g., :string) is entered in a type-in field, then the slot
is treated as an ordinary parameter, and all items in the list will
have a formula that accesses this slot in the aggrelist. For example,
if a list consists of rectangles, and the rectangles should all
have the line-style that is passed to the aggrelist, then
the user would select one of the rectangles and enter the an
appropriate name, such as {\tt :line-style}, next to the :line-style
slot of the rectangle.



\end{itemize}


\section{Arrange}
\index{arrange}
\begin{itemize}
\item \index{bring to front}
{\tt Bring to Front:} Brings the selected objects to the front
of their aggregadget (i.e., they will cover all other objects in their
aggregadget).  If multiple objects are selected, it brings the objects to the
front in their current order.

\item \index{send to back}
{\tt Send to Back:} Sends the selected objects to the back of their
aggregadget (i.e., they will be covered by all other objects in their
aggregadget). If multiple objects are selected, it sends the objects
to the back in their current order.

\item \index{aggregadget, make}
{\tt Make Aggregadget:} Creates a new aggregadget and adds all
selected objects (both primary and secondary selections) to it.
The selected objects must initially belong to the same aggregadget or else
Lapidary will print an error message and abort the operation.
The {\tt :left} and {\tt :top} slots of the
objects added to the aggregadget are constrained to the aggregadget unless
they were already constrained (if the object is a line, the {\tt :x1}, {\tt :y1},
{\tt :x2}, and {\tt :y2} slots are constrained).
The constraints tie the objects to the northwest corner of the aggregadget
and use absolute offsets based on the current position of the objects.
Thus if an object is 10 pixels from the left side of the aggregadget (the
bounding box of the aggregadget is computed from the initial bounding
boxes of the objects), the object's {\tt :left} slot will be constrained to be 10
pixels from the left side of the aggregadget.
If the object is a line, the object's endpoints will be tied to
the aggregadget's northwest corner by absolute fixed offsets.
These constraints cause the objects to move with the aggregadget
when the aggregadgets moves.
If the user wants different constraints to apply, the user can primary
select an object, secondary select the aggregadget, and attach a different
constraint.  The aggregadget derives its width and height
from its children, so the :width and :height slots of the children are not
constrained to the aggregadget.  Because the aggregadget computes
its width and height from its children, it is not permitted to resize
an aggregadget.

\item \index{ungroup}
{\tt Ungroup:} Destroys selected aggregadgets and moves their components to the
aggregadgets' parents.

\end{itemize}


\section{Constraints}
\index{constraints}
\begin{itemize}
\item {\tt Line Constraints:} Brings up the line constraints dialog box
\index{line constraints}
(Figure \ref{line-constraint}).

\item {\tt Box Constraints:} Brings up the box constraints dialog box
(Figure \ref{box-constraint})
\index{box constraints}

\item \index{C32}
{\tt C32:} Brings up C32. Each primary and secondary selection is displayed
in the spreadsheet, and additional Lapidary objects can be displayed using
the {\tt Point to Object} command. While Lapidary is running, only objects in
Lapidary's drawing windows can be displayed in the spreadsheet. Nothing will
happen if the user attempts to execute the {\tt Point to Object} command on
an object which is not in a Lapidary drawing window. The C32 chapter
describes how to use C32 and Section
\ref{custom-constraint} describes a number of modifications Lapidary makes
to C32.

\end{itemize}


\section{Other}
\index{other, menu selection}
\begin{itemize}
\item \index{interactors}
{\tt Interactors:} Displays a menu of interactors that the user
can choose to look at. Once the user selects an interactor, the information
from that interactor will be displayed in the appropriate interactor
dialog box (see Section \ref{interactors}) and the user is free to change it.
In addition, menu items are provided for the five Garnet-defined interactor
types: choice (encompassing both menu and button interactors), move/grow,
two-point, text, and angle.  If the user has selected a set of objects, then the interactors menu will
contain all interactors associated with these objects.
Lapidary will display all interactors whose {\tt :start-where} slot references
these objects, or whose {\tt :feedback-obj} or {\tt :final-feedback-obj} points
at these objects.  If no objects are selected, then the interactors menu will contain all interactors
that have been created in Lapidary.

\item \index{clear  workspace}
{\tt Clear Workspace:} Deletes all objects from Lapidary but does not destroy any
of the drawing windows.

\end{itemize}


\section{Test and Build Radio Buttons}
\begin{itemize}
\item \index{test}
{\tt Test:} Deactivates the Lapidary interactors that operate
on the drawing windows and activates all user-defined interactors. This allows
the user to experiment with the look-and-feel that the user has created.

\item \index{build}
{\tt Build:} Deactivates all user-defined interactors and reactivates the Lapidary
interactors, allowing the user to modify the look-and-feel.
\end{itemize}



\chapter{Creating Constraints}
\index{constraints}
\label{constraints}

Lapidary provides two menus for creating constraints, one that deals with
``box'' constraints (constraints on non-line objects)
and one that deals with line constraints.
In addition, several of the property menus provide a custom constraint option
that allows the user to input a constraint that determines the property.
Each of the menus contains buttons labeled with tiny rectangular boxes that
indicate how an object will be positioned if the constraint associated with
that button is chosen.
The rectangular boxes in the buttons are colored black to indicate that
the primary object is the object that will be constrained, and
the white rectangular
boxes positioned at the four corners of the rectangle in the box constraint
menu indicate that the secondary selection is the object that
will be referenced in the constraint.

The Box and Line Constraint dialog boxes, can be used separately from Lapidary
(see section \ref{constraint-gadget}).

The constraint menus can display the current position and size of a primary
selection. By pressing the {\tt Show Constraints} button in the constraint menus,
the user can see what types of constraints are on the slots of an object.
If two objects are selected, Lapidary will display the types of the
constraints between the two objects.

\section{Box Constraints}
\label{box-constraint-section}

The box constraint menu allows constraints to be attached to the {\tt :left},
{\tt :top}, {\tt :width}, and {\tt :height} of an object (see Figure
\ref{box-constraint}).
The user attaches constraints by first selecting the object to be constrained
(a primary selection) and the object to be referenced in the constraint
(a secondary selection).
The user then selects the appropriate buttons in the box constraint menu.
The possible constraints for the {\tt :left} slot are:


\begin{figure}
\bar{}
\vspace{0.5 line}
\begin{center}
\graphic{Postscript=`lapidary/box-constraint-menu.PS',magnify=0.7,
	boundingbox=File}\end{center}
\caption{
The constraint menu for box-like objects on the left, and a
drawing window on the right. The white
rectangle in the drawing window is the object to be constrained and the gray
rectangle is the object to be referenced in the constraint.
The white rectangle is constrained to
be offset
from the right of the gray rectangle by 20 pixels, and
aligned at the top-inside of the gray rectangle.
The white rectangle's width is not constrained and it is 33\% as tall
as the gray rectangle. If the gray rectangle
changes, the white one will be adjusted automatically.
}
\tag{box-constraint}
\bar{}
\end{figure}

\begin{itemize}
\item left-outside: The right side of the primary selection is aligned with the
left side of the secondary selection.

\item left-inside: The left side of the primary selection is aligned with the
left side of the secondary selection.

\item center: The center of the primary selection is aligned with the center
of the secondary selection.

\item right-inside: The right side of the primary selection is aligned with the
right side of the secondary selection.

\item right-outside: The left side of the primary selection is aligned with the
right side of the secondary selection.

\end{itemize}

The possible constraints for the {\tt :top} slot are:

\begin{itemize}
\item top-outside: The bottom side of the primary selection is aligned with the
top side of the secondary selection.

\item top-inside: The top side of the primary selection is aligned with the
top side of the secondary selection.

\item center: The center of the primary selection is aligned with the center
of the secondary selection.

\item bottom-inside: The bottom side of the primary selection is aligned with the
bottom side of the secondary selection.

\item bottom-outside: The top side of the primary selection is aligned with the
bottom side of the secondary selection.

\end{itemize}

The only option for the {\tt :width} slot is to constrain
the width of the primary selection to the width of the secondary selection
and the only option for the {\tt :height} slot is to constrain the
height of the primary selection to the height of the secondary selection.
In addition, each of the four slots may have a custom constraint attached
to them (see Section \ref{custom-constraint}).
Each of the four slots also has an ``Unconstrain'' option that destroys the
constraint attached to that slot.

The constraints in the box constraint menu can be fine-tuned by entering
offsets for the constraints, and in the case of the size slots
(width and height), scale factors as well.
When an object is centered with respect to another object, the offset
field changes to a percent field denoting an interval where 0\% causes
the center point of the constrained object to be attached to the left or
top side of the object referenced in the constraint and 100\% causes the
center point of the constrained object to be attached to the right or
bottom side of the object referenced in the constraint.
By default this percentage is 50.
The {\tt Difference in pixels} and {\tt Scale} factors cause the width and
height constraints to be computed as {\it Scale * Dimension + Difference in pixels}.

Finally, each of the slots has a labeled box next to its name that allows
the user to type in an integer that will be placed in that slot.
If there is already a constraint in the slot, the constraint will not be
destroyed so the value will only temporarily
override the value computed by the constraint (the next time the constraint
is recomputed, the value will be lost).  This operation works only when
there is one primary selection and no secondary selections.

\section{Line Constraints}
\label{line-constraint-section}

The line constraint menu allows the endpoints of a line to be attached to
other objects or the {\tt :left} and {\tt :top} slots of a box object to be
constrained to the endpoint of a line (Figure \ref{line-constraint}).
The buttons on the box and line object in Figure \ref{line-constraint}
indicate the various locations where the endpoint of a line can be attached
to a box or line object or where a point of
a box can be attached to a line.
Thus the two endpoints of a line can be attached to any of the corners, sides,
or center of a box object and any of the corners, sides, or center of a box
object can be attached to the endpoints or center of a line.

\begin{figure}
\bar{}
\vspace{0.5 line}
\begin{center}
\graphic{Postscript=`lapidary/line-constraint-menu.PS',magnify=0.7,
	boundingbox=File}\end{center}
\caption{
The constraint menu for line-like objects on the left, and a
drawing window on the right.  The arrow in the drawing window is the
object to be constrained and the circle is the
object to be referenced in the constraint.
In the ``obj-to-constrain'' section
of the constraint
menu, the line feedback object has been rotated so that it has the
same orientation as the selected arrow, and the box feedback object
has been disabled (grayed-out). In the ``obj-to-reference'' section
the line feedback object has been disabled since the object to be
referenced in the constraint
is a box-like object. The darkened buttons on the right endpoint of
the line feedback object and the left corner of the box feedback object
indicate that the right endpoint of the arrow is attached to the left
corner of the circle.
}
\tag{line-constraint}
\bar{}
\end{figure}

The line object in the constraint menu is oriented in the same direction
as the selected line in the
drawing window, so that the user knows which endpoint is being constrained.
The buttons with the blackened rectangles indicate the points that can be
constrained in the primary selection.
Similarly, the buttons with the white rectangles indicate the points in the
secondary selection that
the primary selection can be attached to.

The {\tt Unconstrain} button at the bottom of the menu allows the user to
destroy the constraint on the selected point and the {\tt Customize} button
allows the user to input a custom constraint (described in Section
\ref{custom-constraint}). Finally the {\tt x-offset} and {\tt y-offset} labeled
boxes allow the user to enter offsets for the constraint.
All offsets are added to the value computed by the constraint.
For example, an x-offset of 10 causes an endpoint constrained to the northwest
corner of a box object to appear 10 pixels to the right of that corner.

The end points of a line can be set directly by typing in values for
x1, y1, x2 and y2.  This function is only active if there is a single line as
the primary selection and no secondary selections.

\section{Custom Constraints}
\label{custom-constraint}

When the user selects the custom constraint option in any of the constraint or
property menus, Lapidary brings up the C32 spreadsheet and a formula window
for the desired slot. The user should enter a formula and press OK (or
cancel to stop the operation). Both the OK and cancel buttons in the formula
window will make C32 disappear.

Information on C32 can be found in the
C32 chapter. However, Lapidary modifies C32 in a number of ways that
are important for a user to know. First, it generates indirect references
to objects rather than direct references. A direct reference explicitly
lists an object in a constraint, whereas an indirect reference accesses
the object indirectly through a link. For example, {\tt (gv rect1 :left)} is
a direct reference to {\tt rect1}, whereas
{\tt (gvl :link0 :left)} is an indirect reference to {\tt rect1} (this assumes
that a pointer to {\tt rect1} is stored in {\tt :link0}). If the
user always generates
references using the C32 functions {\tt Insert Ref from Mouse} and
{\tt Insert Ref from Spread}, then Lapidary will automatically generate
indirect references and create appropriate link names. The user can
edit these link names by bringing up the parameters menu and hitting
the link parameters button (see Section \ref{parameters}).
However, if the user inserts the references by typing them in, then
the user should take care to use the {\tt gvl} form and create the appropriate
links. When the formula is completed, Lapidary checks whether there are
any direct references in the formula and generates a warning if there are.
At this point the user has the option of editing the formula or continuing
with the formula as is. If the user chooses to leave direct references in
the formula, Lapidary may not be able to generalize it, so the formula may
behave strangely if it is inherited.

The second change Lapidary makes is in copying formulas. Lapidary copies
all the links that the formula references to the object which is receiving
the copied formula. If the
links should point to new objects, the user must manually change them
by selecting the {\tt Show All Slots} option in C32 and editing the
appropriate links (the names of the links that need to be modified can be
found by looking at the formula).


\section{The Constraint Gadget}
\label{constraint-gadget}
\index{constraint-gadget}

The constraint menus have been bundled into a constraint gadget that can
be used independently of Lapidary.
The constraint gadget provides two menus: a box constraint menu for specifying
box-type constraints (Figure \ref{box-constraint})
and a line constraint menu for specifying line-type
constraints (Figure \ref{line-constraint}). These menus operate as
described in Sections \ref{box-constraint-section} and
\ref{line-constraint-section}. The menus also provide
access to C32 through {\tt customize} buttons.
The module can be loaded
independently of Lapidary with
\pr{(garnet-load `lapidary:@{\tt\char`\|}constraint-@{\tt\char`\|}gadget-@{\tt\char`\|}loader')} and
is exported from the gadgets package (garnet-gadgets).

\subsection{Programming Interface}

The constraint gadget can be created (or made visible, if already created)
by executing one of the \pr{-do-go} or \pr{show-} functions described in
section \ref{Functions}.  Certain slots of the gadget, described
in section \ref{cg-parameters}, are then set with the objects to be constrained.
When the user operates the buttons in the dialog box, constraints will be
set up among the indicated objects.


\paragraph{Slots of the Constraint Gadget}
\label{cg-parameters}

The constraint gadget exports one object called \pr{gg::*constraint-gadget*}.
This object contains four settable slots:

\begin{description}
\item[] \pr{:obj-to-constrain} - The object which should be constrained. This
slot expects only one object \dash it will not take a list.

\item[] \pr{:obj-to-reference} - The object which should be referenced in the
constraint. This slot expects only one object \dash it will
not take a list.

\item[] \pr{:top-level-agg} - The top level aggregate containing constrainable
objects. If the aggregate associated with a window is
the top level aggregate, this slot may be left NIL (the
default). However, if, for example, the window contains an editor aggregate
and a feedback aggregate, then the \pr{:top-level-agg} slot should
be set to the editor aggregate.

\item[] \pr{:custom-function} - A function that is executed whenever a constraint
is attached to a slot. The function should
take three parameters: an object, a slot, and a formula. The function
is called after the formula has been installed on the slot, but before
the formula has been evaluated.
{\it This function is not called when the user calls the c32 function and
provides a c32-custom-function as a parameter} (see Section \ref{functions}
for details on the c32-custom-function and its parameters). The function is not
called in this case since the constraint gadget does not install the
formula if the c32-custom-function is provided.
\end{description}

It is also possible to prevent either the box-constraint or line-constraint
menus from attaching a constraint to a
slot by adding the slot's name to a list in the \pr{:do-not-alter-slots}
of an object. For example, to prevent a constraint from being attached
to the \pr{:width} or \pr{:height} slots of a text object, the list
\pr{'(:width :height}) could be placed in the object's
\pr{:do-not-alter-slots} slot.
If the user tries to attach a constraint to that slot, an error box
will be popped up indicating that a constraint cannot be attached to
that slot. C32 does not recognize the \pr{:do-not-alter-slots}, and
therefore the box-constraint and line-constraint menus cannot prevent
the user from inserting a formula into a forbidden slot if a customize
button is chosen.



\paragraph{Exported Functions}

\label{Functions}

The following functions are exported from the constraint gadget module:

\index{box-constraint-do-go}
\index{line-constraint-do-go}
\begin{programexample}
gg:Box-Constraint-Do-Go \value{function}

gg:Line-Constraint-Do-Go \value{function}
\end{programexample}

These functions create the Box and Line Constraint dialog boxes.  They should
{\it not} be executed multiple times, since there is only one
\pr{constraint-gadget} object.  If the user clicks on an ``OK'' button and makes
the dialog boxes invisible, then the following functions can be called to
make them visible again:

\index{show-box-constraint-menu}
\index{show-line-constraint-menu}
\begin{programexample}
gg:Show-Box-Constraint-Menu \value{function}

gg:Show-Line-Constraint-Menu \value{function}
\end{programexample}

These functions make the Box and Line Constraint dialog boxes visible.  They
can only be called after the \pr{-do-go} functions above have been called
to create the dialog boxes.

\begin{group}
\begin{programexample}
gg:C32 \&optional {\it object  slot} \value{function}
       \&key {\it left  top  c32-custom-function  prompt}
\end{programexample}

This function causes c32 to come up, with the {\it object} displayed in the
first panel of the c32 window.  The formula in {\it slot} will be displayed in
c32's formula editing dialog box.  The keyword parameters are as follows:
\end{group}
\vspace{1 line}

\begin{description}
\item[] {\it left}, {\it top} - Controls placement of query box that users
use to indicate that they are done with C32.



{\it c32-custom-function} - A function to be executed when the user hits
the OK button in a formula window in C32. The function should
take three parameters: an object, a slot, and a formula.
If a custom function is provided, the formula will not be
installed in the slot (thus the function in :custom-function will
not be called\dashit must be called explicitly by the {\it c32-custom-function}
if it should be executed). This
gives the {\it c32-custom-function} an opportunity to defer the
installation of the formula. For example, in Lapidary,
the user can create formulas that define the values of
various slots in an interactor, but until the user presses
the ``create-interactor'' or ``modify'' buttons, the
formulas should not be installed. Thus the Lapidary \pr{c32-custom-function}
places the formulas on a queue, but does not install them.

The constraint gadget stores the links that this formula uses
in a meta-slot in the formula called \pr{:links}.
Like the formula,
the links are not installed. That is, the link slots do
not exist (unless another formula already uses them). Because
the links have not been installed, the constraint gadget stores
the links and the objects they point to in another meta-slot
in the formula called \pr{:links-and-objs}. The contents of
this slot have the form (list (cons (link-name object))$^{\hbox{*}}$).
Links that
already exist because another formula uses them will not be
on this list.

The \pr{c32-custom-function} or the application can
install the links by calling
the c32 function \pr{install-links} which takes a formula and
the object that the links should be installed in as arguments
(the object that is passed to \pr{c32-custom-function} is the object
that should be passed to \pr{install-links}). \pr{install-links}
will create the links, and if the link points to an object
that is in the same aggregate as the object containing the
link, \pr{install-links} will create a path to the reference
object and store it in the link
slot. \pr{install-links}
destroys the \pr{:links-and-objs} slot, so the \pr{c32-custom-function} or
application should take care to save the contents of this slot
if they need to make further use of this information.



\item[] {\it prompt} - A text string that should be displayed in the query box
that appears when C32 is invoked.

\end{description}

\begin{group}
\begin{programexample}
gg:CG-Destroy-Constraint {\it object  slot} \value{function}
\end{programexample}

This function destroys a constraint created by the
constraint gadget. Required parameters are an object
and a slot.

\end{group}

% \begin{comment}
% {\it destroy-constraint-support-slots}: destroys the slots in the :links,
% :offset, and :scale slots of a formula (these are the
% standard support slots created by the constraint gadget).
% A slot is destroyed only if the formula is the only formula
% that depends on this slot. Required parameters are:
% 
% 		object: The object which contains the formula.
% 		formula: A formula.
% 
% 		An optional parameter is:
% 
% 		destroy-meta-info-p: If this parameter is true, the meta
% 			slots :links, :offset, and :scale are destroyed in
% 			the formula. This parameter defaults to nil.
% 
% \end{comment}
\begin{group}
\begin{programexample}
gg:Valid-Integer-P {\it gadget string} \value{function}
\end{programexample}

\pr{Valid-integer-p} determines if a string input by a Garnet gadget contains
a valid integer. If it does not, the gadget's original value
is restored and an error message is printed.

\end{group}


\vspace{1 line}
\begin{group}
\index{install-links}
\begin{programexample}
c32::Install-Links {\it formula  obj} \value{function}
\end{programexample}

This function is provided by c32, though it is not exported.  As mentioned
above, it is useful for installing links when a custom function is provided
in c32.  The {\it formula} should have a \pr{:links-and-objs} slot, whose value
should be a list of the form
(({\it link-slot-name  object}) ({\it link-slot-name object}) ...).
The {\it obj} parameter names the object which the links should be installed
in.
\end{group}


\paragraph{Programming with Links}
\index{link slots}

Each constraint contains indirect references to objects rather than
direct references. The set of link names it uses to make these indirect
references is contained in the \pr{:links}
meta-slot of the formula and the name
of the offset slot it uses is contained in the \pr{:offset} meta-slot.
If the formula
involves the width or height slots, there is also a \pr{:scale} meta-slot,
containing the name of the scale slot that the formula uses.
The constraint gadget generates link and offset names by appending the
suffixes \pr{-over} and \pr{-offset} to the name of the slot that is being
constrained. For example, if the left slot is being constrained, the
link name will be \pr{:left-over} and the offset name will be
\pr{:left-offset}.
These slot names are stored in a formula's \pr{:links} and \pr{:offset}
meta-slots.
For width and height slots, scale names are generated by appending the
suffix \pr{-scale} to the slot name. Thus the scale slot for a height constraint
would be named \pr{:height-scale}. When C32 generates link names, it generates
them by appending a number to the prefix \pr{link-}. Thus it generates
links such as \pr{:link-0} and \pr{:link-1}.

% \begin{comment}
% The rationale for storing slot names rather than actual values in a
% formula's meta-slots is as follows:
% 
% \begin{itemize}
\item link slots: Storing the real objects pointed to by the constraint would
% be difficult because of inheritance. When a formula was inherited,
% it would have to change the object it was pointing to. If the names
% of link slots are stored, it can use the link slot to get the real
% object. The link slot presumably will have been made to point to
% the appropriate object. Also, storing the name of the link slot
% makes it easier for an application to change the link names in the
% formula, if the application is so inclined.
% 
% \item offset and scale slots: Storing the names of the offset and scale slots
% serves two purposes: 1) the application can immediately find out
% what the constraint is calling the offset and scale slots and
% change them if necessary (the user may not want to call an offset
% slot, :left-offset); and 2) when an offset or scale changes, the
% change only has to be made in one place rather than two places.
% For feedback purposes, the application can retrieve the offset and
% scale indirectly using the names of these slots.
% 
% \end{comment}
% 
% 
% 
% \subsection{Custom Constraints}
% \label{custom}
% 
% When the user selects the custom constraint option in any of the constraint
% menus, the constraint gadget brings up the C32 spreadsheet and a formula window
% for the desired slot. The user should enter a formula and press OK (or
% cancel to stop the operation). Both the OK and cancel buttons in the formula
% window will make C32 disappear.
% 
% The constraint gadget modifies C32 in a
% number of ways that
% are important for a user to know. First, it generates indirect references
% to objects rather than direct references. A direct reference explicitly
% lists an object in a constraint, whereas an indirect reference accesses
% the object indirectly through a link. For example, \pr{(gv RECT1 :left)} is
% a direct reference to \pr{RECT1}, whereas
% \pr{(gvl :link0 :left)} is an indirect reference to \pr{RECT1} (this assumes
% that a pointer to \pr{RECT1} is stored in \pr{:link0}). If the
% user always generates
% references using the C32 functions \pr{Insert Ref from Mouse} and
% \pr{Insert Ref from Spread}, then the constraint
% gadget will automatically generate
% indirect references and create appropriate link names. The user can
% edit these link names by finding them in the spreadsheet and modifying them.
% However, if the user inserts the references by typing them in, then
% the user should take care to use the \pr{gvl} form and create the appropriate
% links. When the formula is completed,
% the constraint gadget checks whether there are
% any direct references in the formula and generates a warning if there are.
% At this point the user has the option of editing the formula or continuing
% with the formula as is. If the user chooses to leave direct references in
% the formula, the constraint gadget may not be
% able to generalize it, so the formula may
% behave strangely if it is inherited.
% 
% The second change the constraint gadget makes
% is in copying formulas. The constraint gadget copies
% all the links that the formula references to the object which is receiving
% the copied formula. If the
% links should point to new objects, the user must manually change them
% by selecting the \pr{Show All Slots} option in C32 and editing the
% appropriate links (the names of the links that need to be modified can be
% found by looking at the formula).
% 
% 
% \subsection{Feedback}
% 
% The user can determine which constraints are attached to an object by
% selecting the object and an optional second object that the object may be
% constrained to, and then
% selecting the \pr{Show Constraints} option. The appropriate constraint
% buttons will be highlighted and the offset fields set to the correct
% values. If only one object is selected, then all constraints that
% the constraint menu can represent will be shown. For example, the box
% constraint menu would display the constraints on the left, top, width,
% and height slots. If there are two selections, a constrained object and
% a reference object,
% then only the constraints in the constrained object that depend on the
% reference object are shown.
% 
% 
% 
% \chapter{Interactors}
% \index{interactors}
% \label{interactors}
% 
% Lapidary provides a set of dialog boxes that allow a user to define new
% interactors or modify existing ones. To create or modify an interactor,
% select the {\tt Interactors} command from the Lapidary editor menu.
% Lapidary will display a menu
% listing Garnet-defined and user-defined interactors that may be viewed.
% Select the desired interactor and Lapidary will display the appropriate
% interactor dialog box.
% 
% All interactor dialog boxes have a number of standard items, including
% a set of action buttons, a name box, a {\tt :start-where} field, and buttons
% for events. In addition, each dialog box allows the user to set the
% most commonly changed slots associated with that interactor.  Other slots may
% be set using C32 (see section \ref{custom-constraint}).
% 
% The name field allows the user to type in a name for the interactor. The name
% is not used to name the interactor, but instead is converted to a keyword and
% stored in the interactor's :known-as slot.  If the interactor is saved,
% the user-provided name
% will be placed in the name parameter field for create-instance.
% 
% 
% \section{Action Buttons}
% The action buttons permit the following types of operations:
% 
% \begin{itemize}
\item Create Instance: This operation creates an instance of the displayed interactor
% and, if the user has modified any of the slot values, overrides the
% values inherited from this interactor with the modified values.
% In addition, Lapidary examines the {\tt :start-where} field of the
% new interactor and if the
% start-where includes an aggregadget, adds the interactor to the aggregadget's
% behavior slot.
% 
% \item Modify: This operation stores any changes that the user has made to the
% interactor's slots in the interactor.
% 
% \item Destroy: This operation destroys the interactor.
% 
% \item Save: This operation prompts the user for a file name and then writes out
% the interactor.
% 
% \item C32: This operation brings up C32 and displays the interactor in the
% spreadsheet window. The user can then edit any slot in the interactor.
% Any changes the user makes will not be discarded by the {\tt Cancel} button.
% It is generally advisable to bring up the C32 menu only {\it after} the
% interactor has been created. (the one exception to this rule is when
% C32 appears as the result of pressing a formula button.  If the user
% enters a formula in the formula window, the formula will be installed
% in the instance).  Otherwise the user will end up editing
% the prototype for the interactor to be created, instead of the interactor
% itself.  The C32 chapter describes how to operate C32 and Section
% \ref{custom-constraint} describes the modifications Lapidary makes to C32.
% 
% \item Cancel: This operation discards any changes the user has made to the dialog
% box since the last create-instance or modify command.
% 
% \end{itemize}
% 
% \section{Events}
% \index{events}
% Lapidary allows the user to define the start, stop, and abort events of
% an interactor using event cards.  Each card defines one event and a list
% of events can be generated from a deck of cards.
% Each interactor dialog box contains buttons that pop up a window
% for each event that defines a start, stop or abort event.
% A sample event card is shown in Figure \ref{cards}.
% Selecting {\tt Delete this event} will cause this event to be deleted.
% However, Lapidary will not allow you to delete
% an event card if it is the only one that exists.  {\tt Add an event}
% causes a new event to be created.
% {\tt OK} makes the window disappear and generates the event list for the
% desired event.
% 
% \begin{figure}
% \begin{center}
\graphic{postscript = `lapidary/event-cards.PS', BoundingBox = File}\end{center}
% \caption{A sample event card deck}
% \tag{cards}
% \end{figure}
% 
% Any combination of {\tt shift}, {\tt control}, and {\tt meta} can be selected,
% but if the {\tt any modifier} button is selected then the other
% modifier buttons will become unselected.
% The mouse actions and keyboard items are all mutually exclusive, so selecting
% one will cause the previously selected item to be deselected.
% Events like \#{\tt\char`\\}Return can be generated by simply typing ``Return'' in the
% {\tt Specific keypress} box (quotes are not needed).
% 
% \section{:Start Where}
% Every interactor dialog box displays two commonly used start-wheres for an
% interactor and allows the user to select an alternative one using the
% {\tt other} button (Figure \ref{choice-inter}).
% If {\tt other} is selected, a dialog box will appear which lists all
% possible :start-where's. Once the desired start-where is selected, Lapidary
% will incorporate the selected object in the drawing window into the
% start-where if it is appropriate
% (which it is in all cases but {\tt t} and {\tt nil}). If the start-where
% requires a slot (which the {\tt list} start-where's do), Lapidary will
% request the name of a slot.
% 
% If the user wants a type restriction, then pressing the {\tt type restriction}
% button will cause Lapidary to request a type restriction. A type restriction
% can be either an atom (e.g., opal:text) or a list of items
% (e.g., (list opal:text opal:rectangle)). The type restriction button is
% a toggle button so if it is already selected, selecting it again will cause
% the type restriction to be removed. Also, selecting a new start-where will
% cause the type restriction to be removed.
% 
% 
% \section{Formulas}
% \index{formulas}
% Selecting a formula button in any of the interactor dialog boxes causes
% the interactor to be displayed in the C32 spreadsheet window and the
% current value of the slot associated with the formula button to be displayed
% in a C32 formula window. This value can then be edited into a formula. When
% the {\tt OK} button is pressed in the formula window, C32 disappears and
% the formula is batched with the other changes that have been made to
% the interactor since the last {\tt Create Instance} or {\tt Modify} command.
% The formula is not actually installed until the {\tt Create Instance} or
% {\tt Modify} buttons are selected. If the user selects {\tt Cancel} in the
% interactor dialog box, the formula will be discarded. The formula will also
% be discarded if the user selects {\tt Cancel} in the C32 formula window.
% 
% \section{Specific Interactors}
% \subsection{Choice Interactor}
% \index{choice interactor}
% 
% The choice interactor dialog box allows the user to create either a button
% interactor or menu interactor, depending on whether the {\tt menu} or
% {\tt button} radio button is selected (Figure \ref{choice-inter}).
% The other slots that can be set using this dialog box are:
% 
% \begin{figure}
% \begin{center}
\graphic{postscript = `lapidary/choice-inter.PS', BoundingBox = File, magnify = 0.7}\end{center}
% \caption{Choice interactor dialog box}
% \tag{choice-inter}
% \end{figure}
% 
% \begin{itemize}
\item 
% :start-where. If the user selects either {\tt aggregadget of items} or
% {\tt single item} and there is a least one selection in a drawing window
% (it may be either a primary or secondary selection), then start-where's
% with {\tt :element-of} and {\tt :in-box} are generated with the selected
% object.
% 
% 
% \item 
% :feedback-obj. Selecting the radio button associated with {\tt interim feedback}
% will cause the selected object in the Lapidary drawing windows to become
% the interim feedback for this interactor. If this object is constrained to
% one of the objects that satisfies the start-where or to a component of
% one of these objects, Lapidary will automatically
% generalize the constraints so that the object can appear with any of the
% objects in the start-where.
% 
% The user can also use the {\tt by-demo} option to demonstrate interim feedback.
% Lapidary will pop up an
% OK/Cancel box when an object that satisfies the start-where is selected.
% The user can then use the various Lapidary menus to modify this object
% so that it looks as it should when the object's {\tt :interim-selected}
% slot is set.
% Once the desired look is achieved, the user selects OK and
% the changes will be installed so that the object looks like its original
% self when it is not interim selected, and will look like the by-demo copy
% when it is interim selected.
% 
% Lapidary implements the by-demo operation by comparing the values of
% the following slots in the original object and the copied object:
% {\tt :left, :top, :width, :height, :visible, :draw-function,
% :font, :string, :line-style, :filling-style, :x1, :x2, :y1, :y2}.
% 
% The last option the user can choose is {\tt none} in which case nil will
% be stored in the {\tt :feedback-obj} slot. This will not undo the effects
% of a by-demo operation since by-demo also places nil in the {\tt :feedback-obj}
% slot.
% 
% 
% 
% \item :final-feedback. The options for final feedback are identical to those
% for {\tt :feedback-obj}. The by-demo changes
% will appear when the object's {\tt :selected} slot is set to {\tt t}.
% Multiple final
% feedback objects can be created by selecting several objects
% and pressing the final feedback button.
% Lapidary will then bring up C32
% and prompt the designer for a constraint that determines when to use each
% kind of feedback object at run-time.
% 
% \item :final-function. The user can type in the name of a function that should
% be called when the interactor completes.
% 
% \item :how-set. The user can set the {\tt :how-set} slot by selecting a radio
% button or entering numbers in the {\tt increment-by} and (optionally)
% {\tt max value} fields.
% 
% \end{itemize}
% 
% \subsection{Move/Grow Interactor}
% \label{move-grow-sec}
% \index{move/grow interactor}
% 
% The move/grow interactor dialog box (Figure \ref{lapidary-move-inter}) allows
% the user to specify a move/grow interactor. The slots that can be set
% using this dialog box are:
% 
% \begin{itemize}
\item :start-where. If the user selects either {\tt Object to Press Over} or
% {\tt One of This Aggregate} and there is at least one selection in a drawing
% window (it may be either a primary or secondary selection), then start-where's
% with :in-box and :element-of are generated with the selected object.
% 
% \item :line-p. This slot is set by the {\tt Line} and {\tt Box} buttons.
% If a formula is selected, the formula should return {\tt t} if
% the interactor is moving/growing a line, and {\tt nil} if it is moving/growing
% a box object.
% 
% \item :grow-p. This slot is set by the {\tt Grow} and {\tt Move} buttons.
% If a formula is selected, the formula should return {\tt t} if
% the interactor is growing an object, and {\tt nil} if it is moving an object.
% 
% \item :min-length. Specifies a minimum length for lines.
% 
% \item :min-width. Specifies a minimum width for box objects.
% 
% \item :min-height. Specifies a minimum height for box objects.
% 
% \item 
% :obj-to-change. The user can let the move/grow interactor modify the
% object that satisfies the start-where, present an example object
% to change to Lapidary or use a formula to compute the
% object to change.
% This slot would be set if the interaction should start
% over a feedback object such as selection handles, but should actually
% move the object under the feedback object.
% 
% If the user presents an example object to change to Lapidary by
% selecting an object and pressing the {\tt Change this object} button,
% Lapidary will automatically construct
% a formula so that the interactor changes the correct object at
% run-time. For example, in Figure \ref{lapidary-move-inter}, the user
% wants the move interactor to start over one of the selection handles,
% but wants the object highlighted by the selection handles moved. The
% user can specify that the interactor should start over the selection
% handles by selecting the aggregate containing
% the selection handles and pressing the {\tt One of This Aggregate} radio
% button in the move/grow dialog box. The user can specify that the object
% highlighted by the selection handles at run-time should be the object
% changed by selecting the example object
% that the selection handles currently highlight and pressing
% the {\tt Change this object} button in the move/grow dialog box.
% Occasionally Lapidary may not be able to determine
% from the start-where objects which object should be changed at run-time.
% In this case Lapidary will give the user the choice of entering a formula
% or of having the example object selected as the obj-to-change be the actual
% object changed at run-time.
% 
% 
% 
% \item :final-function. The user can type in the name of a function that should
% be called when the interactor completes.
% 
% \item :feedback-obj. An interim feedback object can be created by creating the
% desired object and
% pressing the interim-feedback button.  Constraints will be automatically
% attached to the feedback object that cause it to move or grow appropriately,
% and that make it visible/invisible at the appropriate times.
% If multiple objects are selected,
% Lapidary will bring up C32
% and prompt the designer for a constraint that determines when to use each
% kind of feedback object at run-time.
% 
% \item :attach-point. Controls where the mouse will attach to the object.
% 
% \end{itemize}
% 
% The grow and move parameters allow the user to control which slots in
% the object that is being grown or moved will actually be set. If a
% formula is entered, it must return a value that can be used by the
% slot :slots-to-set (see the Interactors chapter for more details on
% this slot).
% 
% Example: To create an interactor that moves a box,
% 
% \begin{enumerate}
\item Create the box and leave it selected
% 
% \item Select interactors from the editor menu and then select move/grow
% 
% \item In :start-where click on ``Object to Press Over''.  This will
% cause the selected rectangle's KR name to be displayed.
% 
% \item Press the CREATE INSTANCE action button
% \end{enumerate}
% 
% To test the interactor press test button in the editor menu and
% drag it around.
% 
% 
% \begin{figure}
% \bar{}
% % \comment{l, bot, l->r,b->t}
\begin{center}
\graphic{Postscript=`lapidary/move-behavior-a.PS',magnify=0.6,
                  boundingbox=file}\graphic{
                 Postscript=`lapidary/move-inter.PS',boundingbox=file,
                  magnify=0.6}\end{center}
\begin{tabular}
\tabset{1.4375in, 4in}\\
 & (a) & (b)\\
\end{tabular}

\caption{
(a) The various parts for a move behavior; and (b)
the dialog box for specifying the move/grow interactor.
The user can now directly select the obj-to-change and
Lapidary will create a formula that automatically selects
the correct object to change at run-time.
}
\tag{lapidary-move-inter}
\bar{}
\end{figure}


\subsection{Two Point Interactor}
\index{two point interactor}

The two point interactor dialog box (Figure \ref{two-point-inter}) allows
a user to create a two point interactor. The slots that can be set
using this dialog box are:

\begin{itemize}
\item :start-where. If the user selects {\tt Start Anywhere in Window} then a
start-where with {\tt t} is generated. if the user selects
{\tt Start in Box} and there is at least one selection in a drawing
window (it may be either a primary or secondary selection), then a start-where
with :in-box is generated with the selected object.

\item :line-p. This slot is set by the buttons {\tt Create Line} and
{\tt Create Non-Line}. If a formula is selected, the formula should return
{\tt t} if
the interactor is creating a line, and {\tt nil} if it is creating
a box object.

\item :min-length. Specifies a minimum length for lines.

\item :min-width. Specifies a minimum width for box objects.

\item :min-height. Specifies a minimum height for box objects.

\item :flip-if-change-side. Indicates whether a box may flip over when it is
being created.

\item :abort-if-too-small. Indicates whether the operation should be aborted
if the object is too small or whether an object of the minimum size
should be created.

\item :feedback-obj. An interim feedback object
can be created by creating the desired object and pressing the
interim-feedback button.  Constraints will be automatically attached to
the feedback object that cause it to sweep out as the mouse cursor
is moved, and that make it visible/invisible at the appropriate times.
If multiple objects are selected, then
Lapidary will bring up C32
and prompt the designer for a constraint that determines when to use each
kind of feedback object at run-time.
If the standard feedback option is selected, a box or line feedback object
is automatically created according to whether a line or box is being created.

\item :final-function. The user can type in the name of a function that should
be called when the interactor completes.

\end{itemize}

Example: To create a two-point interactor with line feedback

1) select the interactors option from {\tt other} in the editor menu
and then select two point interactor in the menu that pops up

2) click on Start Anywhere in Window

3) click on Create Line

4) click on Standard Feedback

5) click on CREATE INSTANCE

To test this interactor, enter test mode, press down on the left mouse button, and sweep out a line.  No line will be created because a final function
was not provided.

\begin{figure}
\begin{center}
\graphic{postscript = `lapidary/TwoPoint.PS', Magnify = 0.8, BoundingBox = File}\end{center}
\caption{Two point interactor dialog box}
\tag{two-point-inter}
\end{figure}


\subsection{Text Interactor}
\index{text interactor}

The text interactor dialog box (Figure \ref{text-inter})
allows the user to create or modify a
text interactor and to edit the following slots:

\begin{figure}
\begin{center}
\graphic{postscript = `lapidary/text-inter.PS', BoundingBox = File, magnify=0.7}\end{center}
\caption{Text interactor dialog box}
\tag{text-inter}
\end{figure}

\begin{itemize}
\item 
:start-where. If the user selects either {\tt object to press over} or
{\tt one of this aggregadget} and there is at least one selection in a drawing
window (it may be either a primary or secondary selection), then start-wheres
with {\tt :in-box} and {\tt :element-of} are generated with the selected
object.



\item :obj-to-change. The user can either let the text interactor modify the
object that satisfies the start-where or use a formula to compute the
object to change.  Lapidary can construct a formula for this slot if
necessary (see section \ref{move-grow-sec}).

\item :feedback-obj. An interim feedback object
can be created by creating a text object and pressing the
interim-feedback button.  Constraints will be automatically attached to
the feedback object that cause it to appear at the selected text object
and that make it visible/invisible at the appropriate times.
If multiple objects are selected, then
Lapidary will bring up C32
and prompt the designer for a constraint that determines when to use each
kind of feedback object at run-time.

\item :cursor-where-press. This slot is set by the buttons {\tt where pressed}
and {\tt at end of string}. If {\tt where pressed} is selected, the
text editing cursor
will appear under the mouse cursor. If {\tt at end of string} is selected,
the text editing cursor will always appear at the end of the string when
editing starts.

\item :final-function. The user can type in the name of a function that should
be called when the interactor completes.

\end{itemize}

\subsection{Angle Interactor}

The angle interactor dialog box (Figure \ref{angle-inter}) allows the user
to create and modify an angle interactor. The slots that can be set by
this dialog box are:

\begin{itemize}
\item :start-where. If the user selects {\tt object to press over}
and there is a least one selection in a drawing
window (it may be either a primary or secondary selection), then a start-where
with {\tt :in-box} is generated. If the user selects {\tt start anywhere in
window}, then a start-where of {\tt t} is generated.

\item :obj-to-change. The user can either let the angle interactor modify the
object that satisfies the start-where or use a formula to compute the
object to change.

\item :feedback-obj. An interim feedback object can be created by creating
an object, selecting it,
and pushing the interim feedback button.
The :angle slot of the object will be
set as the interactor is operated and the object will be made visible/invisible
as appropriate.  To make the feedback object or the object that gets the final
angle change in response to changes in the :angle slot, custom constraints
must be created for the position and size slots.  See the angle interactor
section in the Interactors chapter for sample constraints.
If multiple objects are selected, then
Lapidary will bring up C32
and prompt the designer for a constraint that determines when to use each
kind of feedback object at run-time.

\item :final-function. The user can type in the name of a function that should
be called when the interactor completes.

\item :center-of-rotation. This is the center of rotation for the interaction.
The user can either enter a list of {\tt (x,y)}, enter a formula that
returns a list of {\tt (x,y)} or select one of the standard locations for
the center of rotation by selecting the appropriate button.

\end{itemize}

\begin{figure}
\begin{center}
\graphic{postscript = `lapidary/Angle.PS', BoundingBox = File, magnify=0.7}\end{center}
\caption{Angle interactor dialog box}
\tag{angle-inter}
\end{figure}


\chapter{Getting Applications to Run}

Lapidary-generated files consist of a set of create-instance
calls.  The objects created are stored in a list and assigned to
the variable *Garnet-Objects-Just-Created*.  The top of a Lapidary-generated
file contains code to load the lapidary-functions.lisp file, which
provides functionality to support the created objects.





