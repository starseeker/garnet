%\make{manual}
%\disable{figurecontents}
%\libraryfile{Garnet}
%\string{TitleString = `Garnet Toolkit'}
%\use{Bibliography = `garnet.bib'}

\documentclass{report}

\usepackage{alltt}
\usepackage{url}
\usepackage{epsfig}
\newenvironment{programexample}{\begin{alltt}}{\end{alltt}}

\begin{document}

\begin{titlepage}
%\begin{titlebox}
% \comment{This is to make sure it does not affect the previous page,
%       e.g. putting it at the top of the file won't cut it:}
  \vspace{0.6 inch} \title{Overview of the Garnet System}
  
  \author{Brad A. Myers \and Andrew Mickish}
%\vspace{0.3 line}
%\value{date}
%\end{titlebox}
%\vspace{0.5 inch}

\begin{abstract}
  This article provides an overview of the Garnet system, and a guide
  to this set of manuals.
\end{abstract}
{\bf Abstract}

\vspace{0.7 inch} \begin{center}
Copyright \copyright{1994} - Carnegie Mellon University
\end{center}

%\vspace{2 lines}
The original Garnet research
was sponsored by NCCOSC under Contract No.
N66001-94-C-6037, Arpa Order No. B326.
The views and conclusions contained in this document are those of
the authors and should not be interpreted as representing the
official policies, either expressed or implied, of NCCOSC or the U.S.
Government.

Modifications Copyright \copyright{2005} by Robert P. Goldman and other
contributors.  All modifications to this document are made freely
available under the terms of the Lesser Gnu Public License.


\end{titlepage}



%%% The following should be redone using fancyheadings.sty
%%% [2003/07/02:rpg]

% \comment{ The \PageHeading command is needed because this manual does not use
%           \Chapter, which normally performs this command. }
%\pageheading{Even,Immediate, left `\underline{Page
%    \value{page}\hsp{0.15 in}\value{TitleString} & }'}


%% I need to figure out how to get the pages numbered contiguously...
%% [2003/07/02:rpg] 
%\string{overview = `1'} % \comment{26 pages}
\string{overview-first-page = `3'}
\string{apps = `27'} % \comment{12 pages}
\string{apps-first-page = `29'}
\string{tour = `41'} % \comment{20 pages}
\string{tour-first-page = `43'}
\string{tutorial = `61'} % \comment{42 pages}
\string{tutorial-first-page = `63'}
\string{kr = `101'} % \comment{52 pages}
\string{kr-first-page = `103'}
\string{opal = `151'} % \comment{70 pages}
\string{opal-first-page = `153'}
\string{inter = `219'} % \comment{78 pages}
\string{inter-first-page = `221'}
\string{aggregadgets = `295'} % \comment{54 pages}
\string{aggregadgets-first-page = `297'}
\string{gadgets = `347'} % \comment{116 pages}
\string{gadgets-first-page = `349'}
\string{debug = `461'} % \comment{20 pages}
\string{debug-first-page = `463'}
\string{demos = `481'} % \comment{10 pages}
\string{demos-first-page = `483'}
\string{sampleprog = `491'} % \comment{14 pages}
\string{sampleprog-first-page = `493'}
\string{gilt = `505'} % \comment{20 pages}
\string{gilt-first-page = `507'}
\string{c32 = `525'} % \comment{12 pages}
\string{c32-first-page = `527'}
\string{lapidary = `537'} % \comment{36 pages}
\string{lapidary-first-page = `539'}
\string{hints = `573'} % \comment{8 pages}
\string{hints-first-page = `575'}
\string{GlobalIndex = `580'}

%\set{page=overview-first-page}
\section{Introduction}
The Garnet research project in the School of Computer Science at
Carnegie Mellon University is creating a comprehensive set of tools
which make it significantly easier to create graphical,
highly-interactive user interfaces.  The lower levels of Garnet are
called the ``Garnet Toolkit,'' and these provide mechanisms that allow
programmers to code user interfaces much more easily.  The higher
level tools allow both programmers and non-programmers to create user
interfaces by just drawing pictures of what the interface should look
like.  Garnet stands for {\it G}enerating an {\it A}malgam of {\it
  R}eal-time, {\it N}ovel {\it E}ditors and {\it T}oolkits.

At the time of this writing, the Garnet toolkit is in use by about 80
projects throughout the world.  This document contains an overview,
tutorial, and a full set of reference manuals for the Garnet System.

{\bf This manual describes version 3.0 of Garnet.  It replaces all
  previous versions and the change documents for versions 1.3, 1.4,
  2.0, 2.1, and 2.2.}

Garnet is written in Common Lisp and can be used with either Unix
systems running X windows or on the Mac.  Therefore, Garnet is quite
portable to various environments.  It works in virtually any Common
Lisp environment, including Allegro (Franz), Lucid, CMU, Harlequin,
CLISP, AKCL, and Macintosh Common Lisps.  The computers we know about
it running on include Sun, DEC, HP, Apollo, IBM RT, IBM 6000, TI, SGI,
NeXTs running X/11, PC's running Linux, Macs, and there may be others.
Currently, Garnet supports X/11 R4 through R6 using the standard CLX
interface.  Garnet does {\it not} use the standard Common Lisp Object
System (CLOS) or any X toolkit (such as Xtk or CLIM).

Garnet has also been implemented using the native Macintosh QuickDraw
and operating system.  To run the Macintosh version of Garnet, you
need to have System 7.0 or later, Macintosh Common Lisp (MCL) version
2.0.1 or later, and at least 8Mb of RAM.  The system takes about 10
megabytes of disk space on a Mac, not including the documentation
(which takes an additional 8 megabytes).  We find that performance of
Garnet on MCL is acceptable on Quadra's, and fine on a Quadra 840 A/V.
It is really too slow on a Mac II.  To do anything useful, you
probably need 12mb of memory.  A PowerPC Mac does not work well for
Lisp (see discussion on \texttt{comp.lang.lisp.mcl}).

More details about Garnet are available in the Garnet FAQ:
\url{ftp://a.gp.cs.cmu.edu/usr/garnet/garnet/FAQ} which is
posted periodically.

This document is a technical reference manual for the entire Garnet
system.  There have been many conference and journal papers about
Garnet (see section \ref{articles} for a complete bibliography).  The
best overview of Garnet is \cite{GarnetIEEE}.  Section \ref{articles}
includes instructions for retrieving some Garnet papers via FTP.

\index{amulet} A new project named Amulet is actively developing a
system with features similar to those in Garnet, but implemented in
C++.  Section \ref{future} discusses how to get more information about
the Amulet project.


\section{Garnet Bulletin Board}
\index{bugs (reporting)} \index{garnet-users} \index{bboard}

There is an international bboard for Garnet users named
\texttt{comp.windows.garnet}.  Topics discussed on this bboard include
user questions and software releases.  There is also a mailing list
called \texttt{garnet-usersjcs.cmu.edu} which carries exactly the same
messages as the bboard.  If you cannot read the bboard in your area,
please send mail to \texttt{garnet-users-requestjcs.cmu.edu} to get on the
mailing list.

You can report bugs to \texttt{garnet-bugsjcs.cmu.edu} which is read only
by the Garnet developers.



\section{Important Features of Garnet}
\index{Features} Garnet has been designed as part of a research
project, so it contains a number of novel and unique features.  Some
of these are:
\begin{itemize}
\item The Lapidary tool is the only interactive tool that allows
  application-specific graphics and new widgets to be created without
  programming.
  
\item The Garnet Toolkit is designed to support the {\it entire} user
  interface of an application; both the contents of the application
  window and its menus and dialog boxes.  For example, Garnet directly
  supports selecting graphical objects with the mouse, moving them
  around, and changing their size.
  
\item It is {\it look-and-feel independent}.  Garnet allows the
  programmer to define a new graphical style, and use that throughout
  a system.  Alternatively, a pre-defined or standard style can be
  used, if desired.
  
\item It uses a {\it prototype-instance} object model instead of the
  more conventional class-instance model, so that the programmer can
  create a {\it prototype} of a part of the interface, and then create
  instances of it.  If the prototype is changed, then the instances
  are updated automatically.  Garnet's custom object system is called
  KR.  Garnet does not use CLOS.
  
\item {\it Constraints} are integrated with the object system, so any
  slot (also called an ``instance variable'') of any object can
  contain a formula rather than a value.  When a value that the
  formula references changes, the formula is re-evaluated
  automatically.  Constraints can be used to keep lines attached to
  boxes, labels centered within rectangles, etc. (see Figure
  \ref{toolkit-SampleFig}).  Constraints can also be used to keep
  application-specific values connected with the values of graphical
  objects, menus, scroll bars or gauges in the user interface.
  
\item Objects are {\it automatically refreshed} when they change.
  Pictures are displayed by creating graphical objects which are
  retained.  If a slot of an object is changed, the system
  automatically redraws the object and any other objects that overlap
  it.  Also, the system handles window refresh requests from X and the
  Mac.
  
\item The programmer specifies the handling of input from the user at
  a high level using abstract {\it interactor} objects.  Typical user
  interface behaviors are encapsulated into a few different types of
  interactors, and the programmer need only supply a few parameters to
  get objects to respond to the mouse and keyboard in sophisticated
  ways.
  
\item There is built-in support for laying out objects in {\it rows
    and columns}, for example, for menus, or in {\it graphs or trees},
  for example, to show a directory structure or a dependency graph.
  
\item Two complete sets of {\it gadgets} (also called widgets or
  interaction techniques) are provided to help the programmer get
  started.  These include menus, buttons, scroll bars, sliders,
  circular gauges, graphic selection, scrollable windows, and arrows.
  One set has the Garnet look and feel, and the other has the Motif
  look and feel.  The Motif set is implemented entirely in Lisp on top
  of Garnet, to provide maximum flexibility.  Note: There are no
  Macintosh look-and-feel gadgets.  When you use Garnet on the
  Macintosh, the gadgets will {\it not} look like standard Macintosh
  widgets.
  
\item Garnet is designed to be {\it efficient}.  Even though Garnet
  handles many aspects of the interface automatically, an important
  goal is that it execute quickly and not take too much memory.  We
  are always working to improve the efficiency, but Garnet can
  currently handle dozens of constraints attached to objects that are
  being dragged with the mouse.
  
\item Garnet will automatically produce PostScript for any picture on
  the screen, so the programmer does not have to worry about printing.
  
\item Gesture recognition (such as drawing an `X' over an object to
  delete it) is supported, so designers can explore innovative user
  interface concepts.
\end{itemize}

%\begin{group}
\section{Coverage}
\index{Coverage} Garnet is designed to handle interfaces containing a
number of graphical objects which the user can manipulate with the
mouse and keyboard.

Garnet is suitable for applications of the following kinds:
%\end{group}
\begin{itemize}
\item User interfaces for expert systems and other AI applications.
  
\item Box and arrow diagram editors like Apple Macintosh MacProject
  (which helps with project management).
  
\item Graphical Programming Languages where computer programs can be
  constructed using icons and other pictures (a common example is a
  flowchart).
  
\item Tree and graph editing programs, including editors for semantic
  networks, neural networks, and state transition diagrams.
  
\item Conventional drawing programs such as Apple Macintosh MacDraw.
  
\item Simulation and process monitoring programs where the user
  interface shows the status of the simulation or process being
  monitored, and allows the user to manipulate it.
  
\item User interface construction tools (Garnet was implemented using
  itself).
  
\item Some forms of CAD/CAM programs.
  
\item Icon manipulation programs like the Macintosh Finder (which
  allows users to manipulate files).
  
\item Board game user interfaces, such as Chess.
\end{itemize}

Figure \ref{toolkit-SampleFig} shows a simple Garnet application that
was created from start to finish (including debugging) in three hours.
The code for this application is shown in the sample program manual,
which begins on page \pageref{sampleprog-first-page}.

\begin{figure}
\centerline{\epsfig{figure=toolkitpic.PS}}
\caption{A sample Garnet application.  The code for this application is
  listed the `Sample Garnet Program' section of this manual, starting
  on page \pageref{sampleprog-first-page}.}  \label{fig:toolkit-SampleFig}
\end{figure}

Other examples of applications created using Garnet appear in the
picture section of this manual, starting on page \pageref{apps}.


%\begin{group}
\section{Running Garnet From /afs}

If you are running Garnet in X windows from CMU, or if you have access
to AFS, you can access Garnet directly on the \texttt{/afs} servers.  We
maintain binaries of the official release version in machine- and
lisp-specific subdirectories of \texttt{/afs/cs/project/garnet/}.  If you
are at CMU, you can skip section \ref{retrieving} altogether, and just
start lisp and load Garnet with:

\begin{alltt}
  (load `/afs/cs/project/garnet/garnet-loader.lisp')
  \end{alltt}

The CMU version of \texttt{garnet-loader.lisp} will attempt to determine
what kind of machine and lisp you're using, and load the appropriate
binaries for you.  You will not have to supply or customize any
pathnames.
%\end{group}

\section{How to Retrieve and Install Garnet}
\label{retrieving}
\index{Getting Garnet} \index{retrieving Garnet} \index{installing
  Garnet} \index{compiling Garnet} \index{Address} \index{License}
\index{FTP Instructions}

Garnet is available for free by anonymous FTP.  There are different
instructions for obtaining the software depending on whether it will
be installed on a Mac or a Unix system (the code is the same, but the
packaging is different).

\subsection{Installation on a Mac}
%\vspace{1 line}

{\bf Retrieving the Stuffit Files:}

Garnet 3.0 is available in \texttt{Stuffit} files that include the
sources, the library files, the binary files compiled for Macintosh
Common Lisp 2.0.1, and documentation.  To download the Garnet
collection that includes MCL binaries, use FTP or Fetch (a Mac file
transfer utility) to connect to \texttt{a.gp.cs.cmu.edu}
(\texttt{128.2.242.7}) and login as \texttt{``anonymous''} with your e-mail
address as the password.  Change to `binary' mode for FTP, or stay in
`automatic' mode for Fetch, and download the \texttt{Stuffit} archive
\texttt{/usr/garnet/garnet/mac.sit} Alternatively, you can get
the BinHex version in text mode by retrieving
\texttt{/usr/garnet/garnet/mac.sit.hqx} If you are using
Fetch, it will automatically convert the BinHex file into a binary
\texttt{.sit} file after it is installed on your Mac.  If you used FTP to
get the \texttt{.hqx} file, you will need to BinHex4 Decode the file.  You
should also retrieve one version of the documentation file:
\begin{alltt}
  /usr/garnet/garnet/doc.sit /usr/garnet/garnet/doc.sit.hqx
\end{alltt}

If you do not have a version of \texttt{Stuffit}, you can also download
the copy of \texttt{Stuffit{\tt\char`\_}Expander} from the same directory
to uncompress the Garnet archive.  The \texttt{Stuffit} utility is a
self-extracting archive that you only need to double-click on to
install on your Mac.  Be sure to use binary transfer mode in FTP if
you are retrieving \texttt{StuffIt{\tt\char`\_}Expander{\tt\char`\_}.sea}.

%\begin{group}
  {\bf Unpacking the Stuffit Files:}
  
  Once you have downloaded the \texttt{.sit} or \texttt{.sit.hqx} archives
  (and installed the \texttt{Stuffit{\tt\char`\_}Expander}, if necessary),
  launch the \texttt{Stuffit} utility.  Next, `Expand...' or `Open' the
  \texttt{mac.sit} archive, and choose a folder into which the
  uncompressed Garnet folder will be expanded.  The instructions below
  assume you have installed the uncompressed folder at the top-level
  of your hard drive, and that your hard drive is named `Macintosh HD'
  (i.e., the uncompressed folder will become `Macintosh HD:Garnet:').
  It is a good idea to expand the \texttt{doc.sit} archive in the Garnet
  folder that was created by the first archive.  For further
  instructions about printing the documentation, consult the README
  file in the doc folder.
%\end{group}
%\vspace{1 line}

{\bf Preparing MCL Before Loading Garnet:}

When using Garnet, you may need to increase the amount of memory that
is claimed by the Lisp application.  You can change the memory claimed
by MCL by selecting the MCL application in the Macintosh Finder and
choosing `Get Info' from the Finder's `File' menu.  Most Garnet
applications will require that MCL use at least 6Mb of RAM, and using
at least 12Mb is recommended.  The default `Preferred size' for MCL is
3072K, so you will need to edit that value to be upwards of 6000K.
You are only allowed to change this information when the application
is NOT running, and it should be done before proceeding with the rest
of these instructions.  Note: All Lisp images saved from the MCL
application will retain the new `Preferred size' value.

Before loading Garnet, you will need to compile several MCL library
files that are used by Garnet.  A compiler script for this procedure
is provided in the Garnet collection.  In the fresh MCL listener, load
the file `Macintosh HD:Garnet:compile-mcl-libraries.lisp' (replacing
the hard drive prefix with whatever is appropriate for your machine).
After the script is finished, quit MCL and then launch MCL again.


{\bf Loading Garnet:}

Using the MCL text editor, edit the file \texttt{garnet-loader.lisp} from
the new Garnet folder (choose `File...', `Open...' from the MCL
menubar to edit a file).  Find the definition of the variable
\texttt{Your-Garnet-Pathname} and set its value to the path of the new
Garnet folder you created with \texttt{Stuffit}.  All other subfolders of
Garnet will be computed relative to this pathname.  Save the new
version of the \texttt{garnet-loader.lisp} file.

In the fresh lisp listener, load `Macintosh
HD:Garnet:garnet-loader.lisp' (using whatever prefix is appropriate
instead of `Macintosh HD:Garnet:').  Garnet will inform you as it
loads each module, and will finally return with a prompt.  At this
point, Garnet is fully loaded and you are ready to try the Tour or
some demos as discussed later in this manual.



\subsection{Installation on a Unix System}

When running on X windows, Garnet uses the CLX interface from Lisp to
X/11.  CLX should be supplied with every Lisp, and the following
instructions assume that CLX has been installed correctly on your
system.  If you need help with CLX, you need to contact your Lisp
vendor.  We cannot help you acquire, compile, or install CLX, sorry.


{\bf Retrieving the TAR Files:}

The Garnet software is about 9 megabytes.  In order to make it easy to
copy the files over, we have created TAR files, so to use the
mechanism below requires double the storage area.  Therefore, you
first need to find a machine with enough room, and then create a
directory called \texttt{garnet} wherever you want the system to be:
\begin{programexample}
  \% mkdir garnet
\end{programexample}

\newpage Then, cd to the \texttt{garnet} directory.
\begin{programexample}
  \% cd garnet
\end{programexample}
Now, ftp to \texttt{a.gp.cs.cmu.edu (128.2.242.7)}.  When asked to log in,
use \texttt{anonymous}, and your name as the password.
\begin{programexample}
  \% ftp a.gp.cs.cmu.edu Connected to A.GP.CS.CMU.EDU.  220
  A.GP.CS.CMU.EDU FTP server (Version 4.105 of 10-Jul-90 12:07) ready.
  Name (a.gp.cs.cmu.edu:bam): anonymous 331 Guest login ok, send
  usernamejnode as password.  Password: 230 Filenames can not have
  '/..' in them.
\end{programexample}

Then change to the \texttt{garnet} directory (note the double
\texttt{garnet}'s) and use binary transfer mode:
\begin{programexample}
  ftp> cd /usr/garnet/garnet/ 
  ftp> bin
\end{programexample}

The files have all been combined into TAR format files for your
convenience.  These will create the appropriate sub-directories
automatically.  We have both compressed and uncompressed versions.
For the regular versions, do the following:
\begin{programexample}
  ftp> get src.tar ftp> get lib.tar
  ftp> get doc.tar
\end{programexample}

To get the compressed version, do the following:
\begin{programexample}
  ftp> get src.tar.Z ftp> get lib.tar.Z
  ftp> get doc.tar.Z
\end{programexample}

Now you can quit FTP:
\begin{programexample}
  ftp> quit
\end{programexample}


{\bf Installing the Source Files:}

If you got the compressed versions, you will need to uncompress them:
\begin{programexample}
  \% uncompress src.tar.Z \% uncompress lib.tar.Z \% uncompress
  doc.tar.Z
\end{programexample}

Now, for each tar file, you will need to `untar' it, to get all the
original files:
\begin{programexample}
  \% tar -xvf src.tar \% tar -xvf lib.tar \% tar -xvf doc.tar
\end{programexample}

This will create subdirectories will all the sources in them.  At this
point you can delete the original tar files, which will free up a lot
of space:
\begin{programexample}
  \% rm *.tar
\end{programexample}

Now, copy the files \texttt{garnet-loader.lisp, garnet-compiler.lisp,
  garnet-@{\tt\char`\|}prepare-@{\tt\char`\|}compile.lisp}, and
\texttt{garnet-after-compile} from the src directory into the \texttt{garnet}
directory:
\begin{programexample}
  \% cp src/garnet-* .
\end{programexample}


{\bf Customizing the PathNames:}

The file \texttt{garnet-loader.lisp} contains variables that should be set
with the pathnames of your Garnet directory and the location of CLX
for your lisp.  You will now need to edit \texttt{garnet-loader.lisp} in
an editor, and set these variables.  Comments in the file will direct
you how to do this.  At the top of the file are the two variables you
will need to set: \texttt{Your-Garnet-Pathname} and
\texttt{Your-CLX-Pathname}.  NOTE: If CLX is already loaded in your lisp
image, you do not need to set the CLX variable.


\newpage {\bf Compiling Garnet to Make Binary Files:}

Lisp requires very large address spaces.  We have found on many Unix
systems, that you need to expand the area that it is willing to give
to a process.  The following commands work in many systems.  Type
these commands to the C shell (csh).  You might want to also put these
commands into your \texttt{.login} file.
\begin{programexample}
  \% unlimit datasize \% unlimit stacksize
\end{programexample}

Now, you will need to compile the Garnet source to make your own
binaries.  This is achieved by loading the compiler scripts.  There is
more information on compiling in section \ref{compilinggarnet} below,
and special instructions for compiling Garnet in CLISP are in section
\ref{clisp}.
\begin{programexample}
  {\it ;; Only LispWorks users need to do the next two commands.  See
    section \ref{lispworks}.}  lisp> \#+lispworks (load
  `src/utils/lispworks-processes.lisp') lisp> \#+lispworks
  (guarantee-processes)
  
  lisp> (load `garnet-prepare-compile') lisp>
  (load `garnet-loader') lisp> (load `garnet-compiler')
\end{programexample}

Now Garnet is all compiled and loaded, but a shell script still needs
to be executed to separate the binary files from the source files.  To
set up for the next time, it is best to quit lisp now, and run the
\texttt{garnet-after-compile} shell script.  If your sources are not in a
directory named \texttt{garnet/src} or your binaries should not be in a
directory named \texttt{garnet/bin}, then you will need to edit
\texttt{garnet-after-compile} to set the directories.  Also, if your
compiler produces binary files that do not have one of the following
extensions, then you need to edit the variable \texttt{CompilerExtension}
in \texttt{garnet-after-compile}: `.fasl', `.lbin', `.sbin', `.hbin',
`.sparcf', `.afasl', or `.fas'.  Otherwise, you can just execute the
file as it is supplied (NOTE: this is run from the shell, not from
Lisp).  You should be in the \texttt{garnet} directory.
\begin{programexample}
  \% csh garnet-after-compile
\end{programexample}

Now you can start lisp again, and load Garnet:
\begin{programexample}
  lisp> (load `garnet-loader')
\end{programexample}

Details about how to customize the loading of Garnet are provided in
section \ref{loading-garnet}.

\section{Directory Organization}
\index{Directories} \index{src} \index{doc} \index{garnet-loader}
\index{garnet-version-number}

All of the information about where various files of Garnet are stored
is in the file \texttt{garnet-loader.lisp}.  This file also defines the
Garnet version number:
\begin{programexample}
  * user::Garnet-Version-Number `3.0'
\end{programexample}

You may need to edit the \texttt{garnet-loader} file to tell Garnet where
all the files are.  Normally, there will be a directory called
\texttt{garnet} with sub-directories called \texttt{src}, \texttt{lib} and
\texttt{bin}.  In the \texttt{src} and \texttt{bin} directions will be
sub-directories for all the parts of the Garnet system:
\begin{itemize}
\item \texttt{utils} - Utility files and functions.
  
\item \texttt{kr} - KR object system.
  
\item \texttt{gworld} - Mac routines for off-screen drawing (only used on
  the Mac)
  
\item \texttt{gem} - Garnet's interface to machine-specific graphics
  routines (X and Mac)
  
\item \texttt{opal} - Opal Graphics management system.
  
\item \texttt{inter} - Interactors input handling.
  
\item \texttt{aggregadgets} - Files to handle aggregates and lists.
  
\item \texttt{gadgets} - Pre-defined gadgets, such as menus and scroll
  bars.
  
\item \texttt{gesture} - Tools for handling gestures as input.
  
\item \texttt{ps} - Functions for printing Garnet windows with PostScript.
  
\item \texttt{debug} - Debugging tools.
  
\item \texttt{demos} - Demonstration programs written using Garnet.
  
\item \texttt{gilt} - The Gilt interface builder.
  
\item \texttt{c32} - A spreadsheet for editing constraints among objects.
  
\item \texttt{lapidary} - The Lapidary interactive tool.
  
\item \texttt{contrib} - Files contributed by Garnet users that are not
  supported by the Garnet group, but just provided for your use.
\end{itemize}

\section{Site-Specific Changes}
\index{Site specific changes} \index{Machine-specific features} If you
are transferring Garnet to your site, you will need to make a number
of edits to files in order for Garnet to load, compile and operate
correctly.  All users will need to edit the Garnet pathnames as
discussed in section \ref{pathnames}, but relatively few users should
need the other sections \ref{optimization-settings} - \ref{clisp}.
Garnet has been adjusted to load on the widest possible variety of
lisps and operating systems with minimum modification.

Of course, if you change any \texttt{.lisp} files in the Garnet
subdirectories (not including \texttt{garnet-loader.lisp}), you will need
to recompile them (section \ref{garnet-load}), even if you do not need
to recompile other parts of Garnet.

\subsection{Pathnames}
\label{pathnames}
\index{pathnames} \index{file names} \index{Garnet-loader}
\index{garnet-version}

After you have copied Garnet to your machine and untar'ed the source
files, the top level Garnet directory will contain the file
\texttt{garnet-loader.lisp}.  This one file contains the file names for
all the parts of Garnet.  You should edit this file to put in your own
file names.  The best way to do this is to set the \texttt{Garnet-Version}
to be \texttt{:external} and edit the string at the top of the file called
\texttt{Your-Garnet-Pathname} to say where the files are.  This change is
normally done during the compile procedure, already described in
section \ref{retrieving}.

\subsection{Compiler Optimization Settings}
\label{optimization-settings}

The variable \texttt{user::*default-garnet-proclaim*}, defined in
\texttt{garnet-loader.lisp}, holds a list of compiler optimization flags
and default values.  These flags determine things like the size and
speed of your resulting Garnet binaries.  For example, the default
value of this variable in Allegro is:

\begin{programexample}
  '(optimize (speed 3) (safety 1) (space 0) (debug 3))
\end{programexample}
%\vspace{-1.5 lines}

This optimization causes Allegro to generate compiled binaries that
are as fast and small as possible.  The {\it safety} setting of 1
means that the compiled code will allow keyboard interrupts if you
somehow go into an infinite loop, and the {\it debug} setting of 3
means you will get the most helpful error messages that Allegro can
give you when you are thrown into the debugger.

Different implementations of lisp require different values for the
optimization flags, and \texttt{garnet-loader.lisp} provides values for
Allegro, Lucid, CMUCL, LispWorks, and MCL that we have found work
particularly well.  You can override the default optimizations by
defining the \texttt{*default-garnet-proclaim*} variable before loading
\texttt{garnet-loader.lisp}.  A value of NIL for this variable means that
you want to maintain the declarations that are already in effect for
your lisp.


\subsection{Fonts in X/11}
\index{Fonts} \index{text-fonts.lisp}

In X/11 R4 through R6, there are almost always a full set of fonts
available with standard names.  Garnet relies on these fonts being
available on the standard font paths set up by X/11.  You can try
loading Garnet and see if it finds the standard fonts.

If not, look in the file \texttt{garnet/src/opal/text-fonts.lisp}.  This
file constructs font names according to the standard X/11 format (with
lots of `-*-*-*''s).  You will have to substitute the names of fonts
that are available at your installation.

\subsection{Keyboard Keys}
\index{Keyboard Keys} \index{Key Caps}

If your keyboard has some specially-labeled keys on it, Garnet will
allow you to use these as part of the user interface.  The file
\begin{programexample}
  define-keys.lisp
\end{programexample}
which is in the \texttt{garnet/src/inter} sub-directory, defines the
mappings from the codes that come back from X/11 and the Mac to the
special Lisp characters or atoms that define the keys in Garnet.

For many machines, such as Suns, HP's, DECStations, and Macs, we have
built in mappings for all of the keyboard keys.  Since there are no
Lisp characters for the special keys, they are named with keywords
such as \texttt{:uparrow} and \texttt{:F1}.  If some keys on your keyboard are
not mapped to keywords, you can use the following mechanism to set
this up.

\index{find-key-symbols} \index{define-keys} To find the correct codes
to use for each undefined key, load the Find-Key-Symbols utility with
\begin{programexample}
  (garnet-load `inter-src:find-key-symbols.lisp')
\end{programexample}
After loading this file, simply type the keys you need to find
mappings for while input is focused on the Find-Key-Symbols window
(you may have to click on the window's title-bar to change the input
focus).  Garnet will print out the code number of the keys you type.

Then, you can go into the file \texttt{define-keys.lisp} and edit it so
the codes you found map to appropriate keywords.

Next, you might want to bind these keys to keyboard editing
operations.  If you want these to be global to all Garnet
applications, then you can edit the files \index{textkeyhandling.lisp}
\texttt{textkeyhandling.lisp} and
\texttt{multifont-textinter.lisp} which contain the default
mappings of keyboard keys to text editing operations.  The Interactors
Manual contains full more information on how this works.

{\bf If you surround your changes to all these files with
  \texttt{\#+{\tt\char`\<}your-switch>} and mail them back to us
  (\texttt{garnetjcs.cmu.edu}), then we will incorporate them into future
  versions so you won't need to continually edit the files.}

\subsection{Multiple Screens}
\index{Garnet-Screen-Number} \index{Multiple Screens} \index{Screens}
If you are working on a machine with only one screen, you need not pay
attention to this section.  However, certain machines, such as the
color Sun 3/60, have more than one screen.  The color Sun 3/60 has
both a black-and-white screen (whose display name is `unix:0.0') and a
color screen (whose display name is `unix:0.1').  If you type `echo
\$DISPLAY' in a Unix shell, you will get the display name of the
screen you are working on; that name should look like `unix:0.*' where
* is some integer.

Garnet assumes that the DISPLAY environment variable has this form of
`displayname:displaynumber.screennumber', and extracts the display and
screen numbers from that.  If any fields are missing, then the missing
display or screen number defaults to zero.


\subsection{OpenWindows Window Manager}
\index{OpenWindows} \index{FocusLenience}

If you are running OpenWindows from Sun, you will need to add the
following line to your \texttt{.Xdefaults} file to make text input work
correctly:
\begin{programexample}
  OpenWindows.FocusLenience: True
\end{programexample}


\subsection{LispWorks}
\label{lispworks}
\index{Harlequin} \index{lispworks} \index{guarantee-processes}

LispWorks is the common lisp sold by Harlequin Ltd.  There is one
peculiarity about LispWorks that requires an additional step before
executing the main-@{\tt\char`\|}event-@{\tt\char`\|}loop background
process of Garnet (Garnet uses multiprocessing by default in LispWorks
-- see the Interactors Manual, section `The Main Event Loop' for
details).  You need to perform this step both when {\it compiling} and
{\it loading} Garnet (the appropriate steps are mentioned during the
standard compile procedure in section \ref{retrieving}).

LispWorks has an unconventional `initialization phaze' to
multiprocessing, which requires that a special function be called
before launching a background process.  There are two ways to
initialize multiprocessing in LispWorks.  One way is to start the big
window-oriented LispWorks interface by executing
\texttt{(tools:start-@{\tt\char`\|}lispworks)}.  This will cause a menu to
appear, and you can open a lisp listener as a selection from the menu.
From this listener, you can load \texttt{garnet-loader.lisp}, and Garnet's
main-event-loop process will be launched by default.

If you do not need all the functionality of the LispWorks interface,
you can initialize multiprocessing with much less overhead.  Before
loading Garnet, load the file \texttt{`src/utils/lispworks-process.lisp'}
and execute the function \texttt{guarantee-processes} to start
multiprocessing.  For example, at the LispWorks prompt you could type:

\begin{programexample}
  {\bf >} (garnet-load `utils-src:lispworks-process.lisp') {\bf >}
  (guarantee-processes)
  
  {\it ;; At this point, a new lisp listener has been spawned} {\bf >}
  (load `garnet/garnet-loader')
\end{programexample}

It is {\it important} to realize that when you call
\texttt{guarantee-processes}, a {\it new} lisp listener is spawned, and
all subsequent commands will be typed into the second listener.
Putting the \texttt{guarantee-processes} call at the top of the
\texttt{garnet-loader.lisp} file will not work, because the first listener
will remain hung at the \texttt{guarantee-processes} call, while the
second process is waiting for user input.

On the other hand, it has been reported that putting the special steps
for LispWorks in a \texttt{.lispworks} file may serve to automate the
process a bit.  To automatically initialize multiprocessing whenever
LispWorks is started, put the following lines in your \texttt{.lispworks}
file:
\begin{programexample}
  (progn (load ``\textit{your-garnet-pathname}/src/utils/lispworks-process.lisp'')
  (guarantee-processes))
\end{programexample}
You will not be able to call \texttt{garnet-load} from your
\texttt{.lispworks} file because the function will not have been defined
when the file is read.

Whenever you enter the debugger of the new listener spawned by
\texttt{guarantee-processes}, you will get restart options that include:

\begin{programexample}
  ...  5 (abort) return to level 0.  6 Return to top level 7 Return
  from multiprocessing
\end{programexample}

When you want to exit the debugger, you should choose either `(abort)
return to level 0,' or `Return to top level', since both of these
options will return you to the top-level LispWorks prompt.  If you
ever choose `Return from multiprocessing', then you will kill both the
second listener and the main-event-loop-process, and you will have to
call \texttt{guarantee-processes} and
\texttt{opal:launch-@{\tt\char`\|}main-@{\tt\char`\|}event-@{\tt\char`\|}loop-@{\tt\char`\|}process}
to restart Garnet's main-event-loop process.

It is not necessary to load \texttt{`lispworks-process.lisp'} or execute
\texttt{guarantee-processes} if you instead choose to execute
\texttt{tools:start-lispworks}.



\subsection{CLISP}
\index{clisp}
\label{clisp}

CLISP is a Common Lisp (CLtL1) implementation by Bruno Haible of
Karlsruhe University and Michael Stoll of Munich University, both in
Germany.  There are a couple of additional steps you must take to run
Garnet in CLISP that are not required in other lisps.


{\bf Renaming .lisp files to .lsp}

If you have an older version of CLISP, you will have to rename all of
the source files from `.lisp' to `.lsp' before starting the procedure
to compile Garnet.  A \texttt{/bin/sh} shell script has been provided to
automate this process in the file \texttt{src/utils/rename-for-clisp}.
This script requires that you \texttt{cd} into the \texttt{src} directory and
execute

\begin{alltt}
\% sh utils/rename-for-clisp
\end{alltt}

The script will rename all of the \texttt{`src/*/*.lisp'} files to
\texttt{`.lsp'}, so that they can be read by CLISP.


{\bf Obtaining CLX}

If you are already using CLISP, you may need to additionally retrieve
the CLX module.  CLX for CLISP can be retrieved via \texttt{ftp} from
\texttt{ma2s2.mathematik.uni-karlsruhe.de}, in the file
\texttt{/pub/lisp/clisp/packages/pcl+clx.clisp.tar.z}.


{\bf Making a Garnet image}

Once you have installed the CLX module, you can make a restartable
image of Garnet with the following procedure (NOTE: this is different
from other lisps).  This is the standard procedure for compiling
Garnet, followed by a dump of the lisp image.

\begin{programexample}
  clisp -M somewhere/clx.mem > (load
  `garnet-prepare-compile.lsp') > (load
  `garnet-loader.lsp') > (load `garnet-compiler.lsp')
  > (opal:make-image `garnet.mem' :quit t)
\end{programexample}

The saved image can then be restarted with the command:

\begin{alltt}
clisp -M garnet.mem
\end{alltt}



\subsection{AKCL}

Some of the default parameters for the AKCL lisp image are
insufficient for running Garnet.  You may be able to change some of
these parameters in the active lisp listener, but it is probably
better to rebuild your AKCL image from scratch with the following
parameter values:

\begin{alltt}
  MAXPAGES for AKCL should be at least 10240, and
  
  (SYSTEM:ALLOCATE-RELOCATABLE-PAGES 800)
  (SYSTEM:ALLOCATE-CONTIGUOUS-PAGES 45 T) (SYSTEM:ALLOCATE 'CONS 3500
  t) (SYSTEM:ALLOCATE 'SYMBOL 450 t) (SYSTEM:ALLOCATE 'VECTOR 150 t)
  (SYSTEM:ALLOCATE 'SPICE 300 t) (SYSTEM:ALLOCATE 'STRING 200 t)
\end{alltt}

Garnet runs about half as fast in AKCL as on other Common Lisps.
Increasing the RAM in your machine may help.  Users have reported that
16MB on a Linux-Box 486 yields unacceptable performance.



%\begin{group}
\section{Mac-Specific Issues}

\subsection{Compensating for 31-Character Filenames:}
There are several gadgets files that normally have names that are
longer than 31 characters.  Mac users may continue to specify the
full-length names of these files by using \texttt{user::garnet-load},
described in section \ref{garnet-load}, which translates the regular
names of the gadgets into their truncated 31-character names so they
can be loaded.  It is recommended that \texttt{garnet-load} be used
whenever any Garnet file is loaded, so that typically long and
cumbersome pathnames can be abbreviated by a short prefix.
%\end{group}

\subsection{Directories:}
Unlike the Unix version, the Macintosh version stores all the binary
and source files together in the various subdirectories under `src'.
This difference will not matter when a Garnet application is moved
between Unix and Mac platforms as long as \texttt{garnet-load} is being
used to load Garnet files.  \texttt{Garnet-load} will always knows where
to find the files.

\subsection{Binding Keys:}
We have assigned Lisp keywords for most of the keys on the Macintosh
keyboard.  Thus, to start an interactor when the `F1' key is hit, use
\texttt{:F1} as the interactor's \texttt{:start-event}.  If you want to know
what a key generates, you can use the small utility
\texttt{Find-Key-Symbols} which has been ported to the Mac.  Execute
\texttt{(garnet-load `inter-src:find-key-symbols')} to bring
up a window which can perceive keyboard events and prints out the
resulting characters.  The data you collect from this utility can be
used in the \texttt{:start-where} slot of interactors to describe events
that will start the interactor, and can be used to modify the
characters generated by the keyboard key by editing the file
\texttt{src:inter:mac-define-keys.lisp}.

\subsection{Simulating Multiple Mouse Buttons With the Keyboard:}
Most of the Garnet demos assume a three button mouse.  To simulate
this on the Macintosh, we use keyboard keys to replace a three-button
mouse.  By default, the keys are \texttt{F13}, \texttt{F14}, and \texttt{F15} for
the left, middle, and right mouse buttons, respectively.  The real
mouse button is also mapped to \texttt{:leftdown}.

You can redefine the keys to be any three keys you want by setting
\texttt{inter::*leftdown-key*}, \texttt{inter::*middledown-key*}, and
\texttt{inter::*rightdown-key*} after loading Garnet or by editing the
file \texttt{src:inter:mac-define-keys.lisp} directly.  These variables
should contain numerical key-codes corresponding to your desired keys.
Some key-codes are shown on p. I-251 of {\it Inside Macintosh Volume
  I}, but you can also do \texttt{(garnet-load `inter:find-key-symbols')}
to run a utility program that tells you the key-code for any keyboard
key.  The utility will generate numbers that can be used directly in
\texttt{src:inter:mac-define-keys.lisp}.

\index{mouse-keys.lisp} To facilitate Garnet's use with keyboards not
equipped with function keys, Garnet supplies another utility program
called \texttt{mouse-keys.lisp}, which is in the top-level Garnet
directory in the Mac version (and is in \texttt{src/utils/mouse-keys.lisp}
if you acquired the Unix-packaged version of Garnet).  When loaded,
this utility creates a window that allows you to toggle between using
the function keys and arrow keys for the simulated mouse buttons.  If
you are frequently switching between using Garnet on an Extended
Keyboard and a smaller laptop keyboard, you may use this utility a lot
to tell Garnet which keys should be used for middle-down and
right-down.

%\begin{group}
\subsection{Modifier Keys:}
Like MCL itself, Garnet treats the \texttt{Option} key as the `Meta' key.
Also, you currently cannot get access to the \texttt{Command} (Open-Apple)
key from Garnet.
%\end{group}

\subsection{Things to Keep in Mind When You Want Your Garnet Programs
  to Run on Both X Windows and the Mac:}
\begin{itemize}
\item Use \texttt{user::garnet-load} instead of \texttt{load} when loading
  gadget files
  
\item Only supply \texttt{:face} values for fonts that run on both systems
  -- this typically restricts you to using only the standard faces
  available in Garnet 2.2 and earlier versions.
  
\item The \texttt{\#+apple} and \texttt{\#-apple} reader macros can be used to
  indicate code that should be used only for Macs and only for
  non-Macs, respectively.  When defining fonts, for example, you may
  want to provide the slot description \texttt{(:face \#+apple :underline
    \#-apple :bold)} to indicate that the font will be underlined on
  the Mac but bold in X.
  
\item The default place for windows is at (0,0) which unfortunately
  puts their title bars under the Macintosh menubar, so you cannot
  even move them using the mouse!  (You can still \texttt{s-value} the
  position from the Lisp Listener.)  Therefore, never create a window
  on the Mac with a \texttt{:top} less than 45 or it will not be movable.
  
\item Remember that many Mac screens are much smaller than most
  workstations' screens.  Positioning windows perfectly may not be
  possible, and a better goal may be to simply keep the window
  title-bars within reach of the mouse so that the windows can be
  moved.
\end{itemize}


\section{Compiling Garnet}
\label{compilinggarnet}
\index{Compiling Garnet} \index{garnet-compiler}

After executing the compile procedure in section \ref{retrieving}, the
result should be that all the files are compiled and loaded.  (If
there was a problem and you need to restart the compile procedure,
please see below.)  The compiler scripts do {\it not} check for
compile errors.  We have attempted to make Garnet compile without
errors on all common lisps, but some lisps generate more warnings than
others.

The compiler scripts compile the binaries into the same directories as
the source files.  For example, all the interactor binaries will be in
\texttt{garnet/src/inter/} along with the source (\texttt{.lisp}) files.
Therefore, after the compilation is completed, you will need to move
the binaries into their own directory (e.g., \texttt{garnet/bin/inter}).
To do this, use the c-shell script
\begin{programexample}
  csh garnet-after-compile
\end{programexample}
The \texttt{garnet-after-compile} file will normally be in the top level
garnet directory.  Note that this is typed to the shell, not to Lisp.
Even if you normally run the `regular' (Bourne) shell (sh), the above
command should work.

To prevent certain parts of Garnet from being compiled, set
\texttt{user::compile-}{\it xxx}\texttt{-p} to NIL, where {\it xxx} is
replaced with the part you do not want to compile.  See the comments
at the top of the file \texttt{garnet-prepare-compile} for more
information.

If you ever have to restart the compile process, you do not have to
start from scratch.  If you have not yet moved the binary files out of
the \texttt{src/} directory (i.e., you have not yet run
\texttt{garnet-after-compile}), then you can use the files that have been
compiled already instead of compiling them again.  Restart lisp, and
for each Garnet module that has been compiled, set the variable
\texttt{user::compile-}{\it xxx}\texttt{-p} to NIL to indicate that it should
not be compiled again.  Then load the three script files again in the
usual order.  Note: if a module has been only partially compiled, then
you must recompile the whole module.


\section{Loading Garnet}
\label{loading-garnet}
\index{Loading Garnet} \index{garnet-loader} \index{load-{\it xxx}-p}
\index{Garnet-{\it xxx}-Pathname} \index{Garnet-{\it xxx}-Src}

To load Garnet, it is only necessary to load the file:
\begin{programexample}
  (load `garnet-loader')
\end{programexample}

(Of course, you may have to preface the file name with the directory
path of where it is located.  It is usually in the top level
\texttt{garnet} directory.)

To prevent any of the Garnet sub-systems from being loaded, simply set
the variable \texttt{user::load-}{\it xxx}\texttt{-p} to NIL, where {\it xxx}
is replaced by whatever part you do not want to load. Normally, some
parts of the system are not loaded, such as the gadgets and demos.
This is because you normally do not want to load or use all of these
in a session.  Files that use gadgets will load the appropriate ones
automatically, and the \texttt{demos-controller} program loads the demos
as requested.

It is possible to save an image of lisp after loading Garnet, so that
when you restart lisp, Garnet will already be loaded and you will not
have to load \texttt{garnet-loader.lisp}.  For details about making lisp
images, see the function \texttt{opal:make-image} in the Opal manual.

%\begin{group}
\section{Loader and Compiler Functions}
\label{garnet-load}
\index{loading} \index{compiling} \index{garnet-load}
\index{garnet-compile}

\subsection{Garnet-Load and Garnet-Compile}

There are two functions that allow you to save a lot of typing when
you load and compile files.  When you supply \texttt{garnet-load} and
\texttt{garnet-compile} with the Garnet subdirectory that you want to get
a file from (e.g., `gadgets'), the functions will automatically append
your Garnet pathname to the front of the specified file.

\begin{programexample}
  user::Garnet-Load `{\it prefix}:{\it filename}'\value{function}
  
  user::Garnet-Compile `{\it prefix}:{\it filename}'\value{function}
\end{programexample}

These functions are defined in \texttt{garnet-loader.lisp} and are
internal to the \texttt{user} package.
%\end{group}

%\vspace{1 line}
%\begin{group}
  The {\it prefix} parameter corresponds to one of the Garnet
  subdirectories, and the {\it filename} is a file in that directory.
  A list of the most useful prefixes appear in section
  \ref{garnet-load-alist}, and a full list can be seen by evaluating
  the variable
  \texttt{user::Garnet-@{\tt\char`\|}Load-@{\tt\char`\|}Alist} in your
  lisp (after loading Garnet).  Examples:

\begin{programexample}
  * (garnet-load `gadgets:v-scroll-loader') Loading
  \#p`/afs/cs/project/garnet/bin/gadgets/v-scroll-loader' Loading
  V-Scroll-Bar...  ...Done V-Scroll-Bar.
  
  T * (garnet-compile `opal:aggregates') Compiling
  \#p`/afs/cs/project/garnet/src/opal/aggregates.lisp' for output to
  \#p`/afs/cs/project/garnet/bin/opal/aggregates.fasl' ...  ; Writing
  fasl file `/afs/cs/project/garnet/bin/opal/aggregates.fasl' ; Fasl
  write complete
  
  NIL *
\end{programexample}
%\end{group}
%\vspace{1 line}

There are two groups of prefixes that \texttt{garnet-load} accepts -- the
`bin' prefixes and the `src' prefixes.  \texttt{Garnet-load} assumes that
when you load files, you will want to load the compiled binaries.
Therefore, when you use prefixes like `gadgets', \texttt{garnet-load} uses
the Garnet-Gadgets-Pathname variable to find the file you want.  If
you really want to load a file from your source directory, you should
use the subdirectory name with `-src' tacked on.  For example,

\begin{programexample}
  * (garnet-load `gadgets-src:motif-parts') Loading
  \#p`/afs/cs/project/garnet/src/gadgets/motif-parts' ...
  
  T *
\end{programexample}

\texttt{Garnet-compile} does not accept `-src' prefixes, because it always
assumes that you want to take a lisp file from your source directory,
compile it, and output it to your bin directory.  Note: do not specify
`.lisp' or `.fasl' with your filename -- \texttt{garnet-compile} will
supply suffixes for you.  \texttt{Garnet-compile} attempts to determine
your correct binary extension (`.fasl', `.lbin', etc.) from the kind
of Lisp that you are using.  If \texttt{garnet-compile} ever gets the
extension wrong, you can change it by setting the variable
\texttt{*compiler-extension*}, which is defined in the \texttt{user} package.

%\begin{group}
\subsection{Adding Your Own Pathnames}
\label{garnet-load-alist}

The functions \texttt{user::garnet-load} and \texttt{user::garnet-compile}
look up their prefix parameters in an association list called
\texttt{user::Garnet-Load-Alist}.  Its structure looks like:

\begin{programexample}
  (defparameter Garnet-Load-Alist `((`opal' . Garnet-Opal-Pathname)
  {\it ; For loading the } multifont-loader (`gg' .
  Garnet-Gadgets-PathName) {\it ; For loading gadgets} (`gestures' .
  Garnet-Gestures-PathName) {\it ; For loading } agate (`debug' .
  Garnet-Debug-PathName) {\it ; For loading the } Inspector (`demos' .
  Garnet-Demos-PathName) {\it ; For loading demos} (`gilt' .
  Garnet-Gilt-PathName) {\it ; For loading high-level tools...}
  (`c32' . Garnet-C32-PathName) (`lapidary' .
  Garnet-Lapidary-PathName) ...))
\end{programexample}

This alist is expandable so that you can include your own prefixes and
pathnames.  Prefixes can be added with the following function:

\index{add-garnet-load-prefix}
\begin{programexample}
  user::Add-Garnet-Load-Prefix {\it prefix pathname} \value{function}
\end{programexample}

For example, after executing \texttt{(add-garnet-load-prefix `home'
  `/usr/amickish/')}, you would be able to do \texttt{(garnet-load
  `home:my-file')}.
%\end{group}



\section{Overview of the Parts of Garnet}
\index{Parts of Garnet} \index{Packages in Garnet}

Garnet is composed of a number of sub-systems, some of which can be
loaded and used separately from the others.  Most of the subsystems
also have their own separate packages.  The following list shows the
components of Garnet, the package used by that component, and the page
number of the corresponding section in this manual.

\begin{itemize}
\item \texttt{KR} - Package \texttt{kr}. \index{kr (package)} The object and
  constraint system.  Page \pageref{kr}.
  
\item \texttt{Gem} - Package \texttt{gem}.  \index{gem (package)} Low-level
  graphics routines that allow the system to run on the Mac or on
  X/11.  We do not support user code directly calling Gem, so it is
  not described further in this manual.
  
\item \texttt{Opal} - Package \texttt{opal}. \index{opal (package)} The
  graphical object system.  Page \pageref{Opal}.
  
\item \texttt{Interactors} - Package \texttt{inter}. \index{inter (package)}
  Handling of mouse and keyboard input.  Page \pageref{Inter}.
  
\item \texttt{Gestures} - Package \texttt{inter}.  Code to handle gesture
  recognition and training.  Described in the interactors manual, page
  \pageref{Inter}.
  
\item \texttt{Aggregadgets} - Package \texttt{opal}.  Support for creating
  instances of collections of objects, and for rows or columns of
  objects.  Page \pageref{aggregadgets}.
  
\item \texttt{AggreGraphs} - Package \texttt{opal}.  Support for creating
  graphs and trees of objects.  Also described in the aggregadgets
  manual, page \pageref{aggregadgets}.
  
\item \texttt{Gadgets} - Package \texttt{garnet-gadgets}, nicknamed \texttt{gg}.
  \index{garnet-gadgets (package)} A collection of pre-defined
  gadgets, including menus, buttons, scroll bars, circular gauges,
  graphics selection, etc.  Page \pageref{gadgets}.
  
\item \texttt{Debugging tools} - Package \texttt{garnet-debug}, nicknamed
  \texttt{gd}.  \index{garnet-debug (package)} Useful functions to help
  debug Garnet programs, including the Inspector.  Page \pageref{Debug}.
  
\item \texttt{Demonstration programs} - Each demonstration program is in
  its own package.  Page \pageref{demos}.
  
\item \texttt{Gilt} - Package \texttt{gilt}. \index{gilt (package)} The Garnet
  interface builder.  Page \pageref{gilt}.
  
\item \texttt{C32} - Package \texttt{c32}.  A spreadsheet interface for
  editing constraints.  Page \pageref{c32}.
  
\item \texttt{Lapidary} - Package \texttt{Lapidary}. \index{lapidary
    (package)} A sophisticated interactive design tool.  Page
  \pageref{lapidary}.
  
\item \texttt{Contrib} - A set of file contributed by Garnet users.  These
  have not been tested by the Garnet group, and are not supported.
  Each file should have a comment at the top describing how it works
  and who to contact for help and more information.
\end{itemize}

\section{Overview of this Technical Report}

In addition to the reference manuals for all the parts of the Garnet
toolkit listed above, this technical report also contains:
\begin{itemize}
\item A guided on-line tour of the Garnet system that will help you
  become familiar with a few of the features of the Garnet toolkit.
  Page \pageref{tour}.
  
\item A tutorial to teach you the basic things you need to know to use
  Garnet.  Page \pageref{tutorial}.
  
\item The code for a simple graphical editor, as a sample of code
  written for Garnet.  Page \pageref{sampleprog}.
  
\item The Hints manual starting on page \pageref{hints} includes some
  suggestions that have been collected from the experience of Garnet
  users for making Garnet programs run faster.  If you have ideas for
  things to add to this section, let us know.
\end{itemize}

\section{What You Need To Know}

Although this is a large technical report, you certainly do not need
to know everything in it to use Garnet.  Garnet is designed to support
many different styles of interface.  Therefore, there are many options
and functions that you will probably not need to use.

In fact, to run the {\it Tour} (page \pageref{tour}), it is not
necessary to read any of the reference manuals.  The tour is
self-explanatory.

Next, you should probably read the {\it Tutorial} (page
\pageref{tutorial}), since it tries to provide enough information about
most of Garnet so that you don't need the other manuals right away.

To run the Gilt Interface Builder, you do not need to know about the
rest of the system either.  The Gilt manual should be sufficient.
When you are ready to set some properties of the gadgets, you will
need to look up the particular gadget in the Gadgets manual to see
what the properties do.

Even when you are ready to start programming, you will still not need
most of the information described here.  To start, you should probably
do the following:
\begin{enumerate}
\item Read this overview.
  
\item Run the tour, to get a feel for Garnet programming.
  
\item Read the tutorial.
  
\item You might try creating a few dialog boxes using Gilt.  This will
  familiarize you with the Gadgets.  See the Gilt manual (Page
  \pageref{gilt}).
  
\item After that, you can look at the sample program at the end of
  this technical report, to see what you need more information about.
  
\item You could now try to start writing your own programs, and just
  use the rest of the manuals as reference when you need information.
  
\item Next, look at the introduction and the following functions in
  the KR document: \texttt{gv, gvl, s-value, formula, o-formula,} and
  \texttt{create-instance}.  The KR manual documents the entire KR module,
  but Garnet does not use every feature that KR provides.  Some
  concepts (like demons), will never be used by the typical Garnet
  user.  Once you have gained some familiarity with the system, you
  may want to return to the KR Manual and read about object-oriented
  programming, type-checking, and constants.
  
\item Next, skim the first five chapters of the Opal manual, and look
  at the various graphical objects, so you know what kinds are
  provided.  The primary functions you will use from Opal are:
  \texttt{add-component, update,} and \texttt{destroy}, as well as the various
  types of graphical objects (\texttt{rectangle, line, circle}, etc.),
  drawing styles (\texttt{thin-line, dotted-line, light-gray-fill}, etc.)
  and fonts.
  
\item Next, in the Interactors manual, you will need to skim the first
  four chapters to see how interactors work, and then see which
  interactors there are in the next chapter.  You will probably not
  need to take advantage of the full power provided by the interactors
  system.
  
\item Aggregadgets and Aggrelists are very useful for handling
  collections of objects, so you should read their manual.  They
  support creating instances of groups of objects.
  
\item You should then look at the gadget manual to see all the
  built-in components, so you do not have to re-invent what is already
  supplied.
  
\item User interface code is often difficult to debug, so we have
  provided a number of helpful tools.  The Inspector is mentioned
  briefly in the Tutorial, and it is discussed thoroughly in the
  debugging manual.  You will probably find many debugging features
  very useful.
  
\item The demo programs can be a good source of ideas and coding
  style, so the document describing them might be useful.
\end{enumerate}

If all you want Garnet for is to display menus and gauges that are
supplied in the gadget set, you can probably just read the KR, Gadgets
and Gilt manuals, and skip the rest.


\section{Planned Future Extensions}
\index{Future work}
\label{future}

We expect 3.0 to be the last release of the lisp version of Garnet.
No enhancements of the lisp version are planned.  However, if you need
something and would like to sponsor its development, write to
\texttt{garnetjcs.cmu.edu}.

\index{amulet} The group is now working on a C++ system called Amulet,
which will have many features similar to those found in Garnet.  Watch
for announcements about the Amulet project on \texttt{comp.windows.garnet}
and \texttt{comp.lang.c++}.  To sign up for the new Amulet mailing list,
please send mail to \texttt{amulet-users-requestjcs.cmu.edu}.


\newpage
\section{Garnet Articles}
\label{articles}
\index{articles} \index{papers}

A number of articles about Garnet have been made available for FTP
from the directory \texttt{/usr/garnet/garnet/doc/papers/} on
\texttt{a.gp.cs.cmu.edu}.  There is a README file in that directory,
indicating which \texttt{.ps} files correspond to the Garnet bibliography
citations.

The following is a complete list of articles written about Garnet, as of
September 4, 1994.  There are 40 refereed journal articles, 4 book chapters,
17 tech reports, and 3 articles submitted for publication.


%\vspace{1 line}
{\bf Refereed Journal Articles}

Brad A. Myers. ``User Interface Software Tools.''
{\it ACM Transactions on Computer-Human Interaction}.  To appear.

Brad Vander Zanden, Brad A. Myers, Dario Giuse and Pedro Szekely.
``Integrating Pointer Variables into One-Way Constraint Models,''
{\it ACM Transactions on Computer-Human Interaction}.  Vol. 1, no. 2,
To appear.

Brad A. Myers.  ``Challenges of HCI Design and Implementation,''
{\it ACM Interactions}.  Vol. 1, no. 1.  January, 1994.  pp. 73-83.

Robert J.K. Jacob, John J. Leggett, Brad A. Myers, and Randy Pausch.
``Interaction Styles and Input/Output Devices,'' {\it Behaviour and
Information Technology}.  March-April, 1993. Vol. 12, no. 2.  pp. 69-79.

Dan R. Olsen, Jr., James D. Foley, Scott E. Hudson, James Miller, and
Brad Myers.  ``Research Directions for User Interface Software Tools,''
{\it Behaviour and Information Technology}.  March-April, 1993. Vol. 12,
no. 2. pp. 80-97.

Brad A. Myers and Brad Vander Zanden.  ``Environment for Rapid
Creation of Interactive Design Tools,'' {\it The Visual Computer;
International Journal of Computer Graphics}. Vol 8, No. 2, February,
1992.  pp. 94-116.

Brad A. Myers, Dario A. Giuse, Roger B. Dannenberg, Brad Vander
Zanden, David S. Kosbie, Ed Pervin, Andrew Mickish, and Philippe
Marchal.  ``Garnet; Comprehensive Support for Graphical,
Highly-Interactive User Interfaces,'' {\it IEEE Computer}. Vol. 23, No.
11. November, 1990. pp. 71-85.  Translated into Japanese and reprinted
in {\it Nikkei Electronics}, No. 522, March 18, 1991, pp. 187-205.

Brad A. Myers. ``A New Model for Handling Input,'' {\it ACM Transactions
on Information Systems}.  Vol. 8, No. 3. July, 1990.  pp. 289-320.

Brad Vander Zanden, Brad A. Myers, Dario Giuse and Pedro Szekely.
``Integrating Pointer Variables into One-Way Constraint Models,''
{\it ACM Transactions on Computer-Human Interaction}.  Vol. 1, no.
2, June, 1994. pp. 161-213.



%\vspace{1 line}
{\bf Refereed Conference Articles}

Brad A. Myers. ``The Garnet User Interface Development Environment:
Demonstration Abstract.'' {\it CHI'94 Conference Companion}.  Boston,
MA, Apr. 24-28, 1994. pp. 25-26.

Brad A. Myers and Dan R. Olsen, Jr. ``User Interface Tools: Tutorial
Description'' {\it CHI'94 Conference Companion}.  Boston,
MA, Apr. 24-28, 1994. pp. 421-422.

Brad A. Myers, Dario Giuse, Andrew Mickish, Brad Vander Zanden, David
Kosbie, Richard McDaniel, James Landay, Matthew Goldberg, and Rajan
Parthasarathy. ``The Garnet User Interface Development Environment:
Video Abstract,'' {\it CHI'94 Conference Companion}.  Boston, MA, Apr.
24-28, 1994. pp. 455-456.

Gurminder Singh, Mark Linton, Brad A. Myers, and Marti Szczur. ``From
Research Prototypes to Usable, Useful Systems: Lessons Learned in the
Trenches,'' {\it Proceedings ACM Symposium on User Interface Software
and Technology: UIST'93}. Atlanta, GA, Nov 3-5, 1993.  pp. 139-143.

Brad A. Myers, Richard G. McDaniel, and David S. Kosbie.  ``Marquise:
Creating Complete User Interfaces by Demonstration,'' {\it Proceedings
INTERCHI'93: Human Factors in Computing Systems}.  Amsterdam, The Netherlands,
April 24-29, 1993.  pp. 293-300.

Brad A. Myers, Richard Wolf, Kathy Potosnak, and Chris Graham.
``Heuristics in Real User Interfaces,''  {\it Proceedings
INTERCHI'93: Human Factors in Computing Systems}.  Amsterdam, The
Netherlands, April 24-29, 1993.  pp. 304-307.

James A. Landay and Brad A. Myers. ``Extending an Existing User Interface
Toolkit to Support Gesture Recognition.'' {\it Adjunct Proceedings of
INTERCHI'93}. Amsterdam, The Netherlands, April 24-29, 1993.  pp. 91-92.

Brad A. Myers. ``The Garnet Gilt Interface Builder: Graphical Styles
and Tabs and Techniques for Reducing Call-Back Procedures,''
Application Builder Session, {\it Seventh Annual X Technical
Conference}, Boston, Massachusetts, January 18, 1993.

Osamu Hashimoto and Brad A. Myers. ``Graphical Styles For Building
User Interfaces by Demonstration,'' {\it ACM Symposium on User Interface
Software and Technology}, Monterey, CA, Nov. 16-18, 1992. pp. 117-124.

Brad A. Myers, Dario A. Giuse, and Brad Vander Zanden. ``Declarative
Programming in a Prototype-Instance System: Object-Oriented
Programming Without Writing Methods,'' {\it Proceedings OOPSLA'92: ACM
Conference on Object-Oriented Programming Systems, Languages, and
Applications}.  October 18-22, 1992.  Vancouver, BC, Canada.
{\it SIGPLAN Notices,} vol. 27, no. 10. pp. 184-200.

Brad A. Myers and Mary Beth Rosson.  ``Survey on User Interface
Programming,'' {\it Proceedings SIGCHI'92: Human Factors in Computing
Systems}.  Monterey, CA, May 3-7, 1992.  195-202.

Brad A. Myers. ``Separating Application Code from Toolkits:
Eliminating the Spaghetti of Call-Backs,'' {\it ACM Symposium on User
Interface Software and Technology}, Hilton Head, SC, Nov. 11-13, 1991.
pp. 211-220.

Brad Vander Zanden, Brad A. Myers, Dario Giuse and Pedro Szekely.
``The Importance of Indirect References in Constraint Models,'' {\it ACM
Symposium on User Interface Software and Technology}, Hilton Head, SC,
Nov. 11-13, 1991. pp. 155-164.

Brad A. Myers.  ``Graphical Techniques in a Spreadsheet for Specifying
User Interfaces,'' {\it Proceedings SIGCHI'91: Human Factors in
Computing Systems}.  New Orleans, LA.  April 28-May 2, 1991.  pp.
243-249.

Brad Vander Zanden and Brad A. Myers, ``Automatic, Look-and-Feel
Independent Dialog Creation for Graphical User Interfaces,''
{\it Proceedings SIGCHI '90: Human Factors in Computing Systems}.
Seattle, WA, April 1-5, 1990. pp. 27-34.

Brad A. Myers.  ``Making it Easy to Create Highly-Interactive,
Graphical Applications in Lisp,'' {\it XNextEvent: The Official
Newsletter of XUG, The X User's Group}.  Vol. 3., No. 1. April, 1990.

Brad A. Myers. ``An Object-Oriented, Constraint-Based, User Interface
Development Environment for X in CommonLisp,'' {\it Fourth Annual X
Technical Conference}, Boston, Massachusetts, January 15-17, 1990.

Dario Giuse. ``Efficient Frame Systems,'' {\it Lecture Notes in
Artificial Intelligence, EPIA 1989}.  Vol. 390. J.P. Martins and E.M.
Morgado, eds.  Springer-Verlag, Sep, 1989.

Brad A. Myers, Brad Vander Zanden, and Roger B. Dannenberg. ``Creating
Graphical Interactive Application Objects by Demonstration,'' {\it ACM
Symposium on User Interface Software and Technology}, Williamsburg,
VA, Nov. 13-15, 1989. pp. 95-104.

Charles Wiecha, Stephen Boies, Mark Green, Scott Hudson, and Brad
Myers.  ``Direct Manipulation or Programming: How Should We Design
Interfaces?''  {\it ACM Symposium on User Interface Software and
Technology}, Williamsburg, VA, Nov. 13-15, 1989. pp. 124-126.

Brad A. Myers. ``AI In Demonstrational User Interfaces,'' {\it A New
Generation of Intelligent Interfaces: IJCAI-89 Workshop}, Detroit, MI.
August 22, 1989, pp. 84-91.

Dario Giuse.  ``Efficient Knowledge Representation Systems,''
{\it The Knowledge Engineering Review}, Vol 4, no. 4, 1990.

Brad A. Myers. ``Encapsulating Interactive Behaviors,''
{\it Proceedings SIGCHI '89: Human Factors in Computing Systems}.
Austin, Texas, April 30 - May 4, 1989, pp. 319-324.

Brad A. Myers. ``User Interface Tools: Introduction
and Survey,'' {\it IEEE Software}, Vol. 6, no. 1, Jan, 1989. pp. 15-23. To
be reprinted in {\it Milestones in Software Evolution}, Paul W. Oman and
Ted G. Lewis, ed.  IEEE Computer Society Press.

Pedro Szekely and Brad Myers. ``A User Interface Toolkit Based on
Graphical Objects and Constraints,'' {\it OOPSLA '88: Conference on
Object-Oriented Programming: Systems, Languages and Applications},
San Diego, CA, September 25-30, 1988.
{\it Sigplan Notices}, Vol 23, no. 11, November, 1988. pp. 36 - 45.

Daniel Kuokka and Dario Giuse. ``The Dante Application Interface,''
{\it Proceedings of the 2nd IEEE International Conference on Computer
Workstations}, Clara, California, March 7-10, 1988.

Dario Giuse.  ``Lisp as a Rapid Prototyping Environment: The Chinese Tutor.''
In {\it Lisp and Symbolic Computation - An International Journal},
Kluwer Academic Publishers, May 1987.


%\vspace{1 line}
{\bf Refereed Published Videotapes}

Brad A. Myers, Dario Giuse, Andrew Mickish, Brad Vander Zanden, David
Kosbie, Richard McDaniel, James Landay, Matthew Goldberg, and Rajan
Parthasarathy. {\it The Garnet User Interface Development Environment}.
Technical Video Program of the CHI'94 conference.
{\it SIGGRAPH Video Review}, Issue 97, no. 13.

Brad A. Myers, Andrew Mickish and Osamu Hashimoto.  ``The Garnet Gilt
Interface Builder: Graphical Styles and Tabs and Techniques for
Reducing Call-Back Procedures,'' Application Builder Video Session,
{\it Seventh Annual X Technical Conference}, Boston, Massachusetts,
January 18, 1993.  10 minutes.

Brad Vander Zanden and Brad A. Myers. {\it Creating Graphical
Interactive Application Objects by Demonstration: The Lapidary
Interface Design Tool.} 12 minute videotape.  Technical Video Program
of the SIGCHI'91 conference, New Orleans, LA.  April 28-May 2, 1991.
{\it SIGGRAPH Video Review}, Issue 64, no. 1.

Brad A. Myers. {\it Some of the Widgets}.  17 minute videotape.
Technical Video Program of Interact'90.  Cambridge, England. August
27-31, 1990.

Brad A. Myers. {\it All the Widgets}.  2 hour videotape.  Technical
Video Program of the SIGCHI'90 conference, Seattle, WA.  April 1-4,
1990.  {\it SIGGRAPH Video Review}, Issue 57.



%\vspace{1 line}
{\bf Chapters in Books}


Brad A. Myers. ``User Interface Software,'' {\it Encyclopedia of Computer
Science and Technology}.  Allen Kent and James G. Williams, editors.
New York: Marcel Dekker, Inc., 1994. To appear.

Brad A. Myers. ``State of the Art in User Interface Software Tools,''
{\it Advances in Human-Computer Interaction, Volume 4}.
Edited by H. Rex Hartson and Deborah Hix.  Norwood, NJ: Ablex
Publishing, 1993.  pp. 110-150.

Brad A. Myers.  ``Garnet: Uses of Demonstrational Techniques,''
{\it Watch What I Do: Programming by Demonstration}, Allen Cypher, et. al., eds.
Cambridge, MA: The MIT Press, 1993.  pp. 219-236.

Brad A. Myers.  ``Ideas from Garnet for Future User Interface
Programming Languages,''
{\it Languages for Developing User Interfaces}. Boston: Jones and Bartlett,
1992.  pp. 147-157.


%\vspace{1 line}
{\bf Technical Reports:}


Brad A. Myers.  {\it User Interface Software Tools}. Carnegie Mellon University
School of Computer Science Technical Report, no. CMU-CS-94-182 and
Human Computer Interaction Institute Technical Report CMU-HCII-94-107.
August 1994.

James A. Landay and Brad A. Myers.  {\it Interactive Sketching for the
Early Stages of User Interface Design}.  Carnegie Mellon University
School of Computer Science Technical Report, no. CMU-CS-94-176 and
Human Computer Interaction Institute Technical Report CMU-HCII-94-104.
July 1994.

Brad A. Myers, Dario A. Giuse, Andrew Mickish, and David S. Kosbie.
{\it Making Structured Graphics and Constraints Practical for
Large-Scale Applications}.  Carnegie Mellon University School of
Computer Science Technical Report, no. CMU-CS-94-109 and Human
Computer Interaction Institute Technical Report CMU-HCII-94-100.  May
1994.

Brad A. Myers.  {\it Why are Human-Computer Interfaces
Difficult to Design and Implement?}  Carnegie Mellon University School
of Computer Science Technical Report, no. CMU-CS-93-183.  July 1993.

Brad A. Myers, editor.  {\it The Second Garnet Compendium:
Collected Papers, 1990-1992}.  Carnegie Mellon University School of
Computer Science Technical Report, no. CMU-CS-93-108, February, 1993.

Brad A. Myers, Dario Giuse, Andrew Mickish, Brad Vander Zanden, David
Kosbie, James A. Landay, Richard McDaniel, Rajan Parthasarathy,
Matthew Goldberg, Roger B. Dannenberg, Philippe Marchal, Ed Pervin.
{\it The Garnet Reference Manuals}.  Carnegie Mellon University Computer
Science Department Technical Report, no. CMU-CS-90-117-R5, Sep. 1994.
Revised from CMU-CS-90-117-R4, Oct. 1993, CMU-CS-90-117-R3, Nov. 1992,
CMU-CS-90-117-R2, May 1992,
CMU-CS-90-117-R, June 1991, CMU-CS-90-117, March, 1990, and
CMU-CS-89-196, Nov. 1989.

Bonnie E. John, Philip L. Miller, Brad A. Myers, Christine M.
Neuwirth, and Steven A. Shafer, eds. {\it Human-Computer Interaction in
the School of Computer Science}.  Carnegie Mellon University School of
Computer Science Technical Report, no. CMU-CS-92-193, October, 1992.

Brad A. Myers. {\it State of the Art in User Interface Software Tools}.
Carnegie Mellon University School of Computer Science Technical
Report, no. CMU-CS-92-114, February, 1992.

Brad A. Myers and Mary Beth Rosson.  {\it Survey on User Interface
Programming}.  Carnegie Mellon University School of Computer Science
Technical Report, no. CMU-CS-92-113, February, 1992.  Also published
as IBM Research Report RC17624.

Brad A. Myers, editor.  {\it The Garnet Compendium: Collected Papers,
1989-1990}.  Carnegie Mellon University School of Computer Science
Technical Report, no. CMU-CS-90-154, August, 1990.

David Kosbie, Brad Vander Zanden, Brad A. Myers, and Dario Giuse.
``Automatic Graphical Output Management,'' {\it The Garnet Compendium:
Collected Papers, 1989-1990}, Carnegie Mellon University School of
Computer Science Technical Report, no. CMU-CS-90-154, August, 1990,
pp. 29-43.

Roger B. Dannenberg, Brad A. Myers, Dario Giuse, and David Kosbie.
``Using Aggregates as Prototypes,'' {\it The Garnet Compendium:
Collected Papers, 1989-1990}, Carnegie Mellon University School of
Computer Science Technical Report, no. CMU-CS-90-154, August, 1990,
pp. 79-93.

Dario Giuse. {\it KR: Constraint-Based Knowledge Representation}.
Carnegie Mellon University Computer Science Department Technical Report,
no. CMU-CS-89-142, Apr, 1989.

Brad A. Myers. {\it The Garnet User Interface Development Environment; A
Proposal}.  Carnegie Mellon University Computer Science Department
Technical Report, no. CMU-CS-88-153, Sept, 1988.

Brad A. Myers. {\it Tools for Creating User Interfaces: An Introduction
and Survey,} Carnegie Mellon University Computer Science Department
Technical Report, no. CMU-CS-88-107, Jan, 1988.

Dario Giuse. {\it KR: an Efficient Knowledge Representation System}.
Carnegie Mellon University Robotics Institute Technical Report, no.
CMU-RI-TR-87-23, Oct, 1987.


%\vspace{1 line}
{\bf Submitted for Publication}

James Landay and Brad A. Myers.  ``Interactive Sketching for the Early
Stages of User Interface Design.''

Nobuhisa Yoda and Brad A. Myers.  ``A High Level Architecture with
Reusable Components for Synchronous Groupware Applications.''

Brad Vander Zanden and Brad A. Myers.  ``The Lapidary Graphical
Interface Design Tool.''





\end{document}