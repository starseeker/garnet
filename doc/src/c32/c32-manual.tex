\device{postscript}
\make{manual}
\disable{figurecontents}
\libraryfile{Garnet}
\string{TitleString = `C32'}
\use{Bibliography = `garnet.bib'}
\modify{FigureCounter, within unnumbered}
\modify{captionenv, fill, Spaces=Compact, below=0}

\define{programinlist=programexample, LeftMargin +0.5inch, RightMargin 0,
	above 0, below 0}


\begin{titlepage}
\begin{titlebox}
\vspace{0.6 inch}
\bg{C32 Reference Manual:
A Constraint Editor}


{\bf Dario Giuse}
\vspace{0.3 line}
\value{date}
\end{titlebox}
\vspace{0.5 inch}
\begin{center}
{\bf Abstract}\end{center}
\begin{text}
C32 is an object and constraint editor for Garnet objects.  It allows
Garnet objects to be viewed and edited using a spreadsheet-like interface.
New values can be installed in the slots of an object directly, or constraints
can be defined among the objects.
\vspace{0.5 inch}
\begin{center}
Copyright \copyright{1994} - Carnegie Mellon University
\end{center}

%\vspace{2 lines}
The original Garnet research
was sponsored by NCCOSC under Contract No.
N66001-94-C-6037, Arpa Order No. B326.
The views and conclusions contained in this document are those of
the authors and should not be interpreted as representing the
official policies, either expressed or implied, of NCCOSC or the U.S.
Government.

Modifications Copyright \copyright{2005} by Robert P. Goldman and other
contributors.  All modifications to this document are made freely
available under the terms of the Lesser Gnu Public License.


\end{text}
\end{titlepage}


\string{overview = `1'} % \comment{26 pages}
\string{overview-first-page = `3'}
\string{apps = `27'} % \comment{12 pages}
\string{apps-first-page = `29'}
\string{tour = `41'} % \comment{20 pages}
\string{tour-first-page = `43'}
\string{tutorial = `61'} % \comment{42 pages}
\string{tutorial-first-page = `63'}
\string{kr = `101'} % \comment{52 pages}
\string{kr-first-page = `103'}
\string{opal = `151'} % \comment{70 pages}
\string{opal-first-page = `153'}
\string{inter = `219'} % \comment{78 pages}
\string{inter-first-page = `221'}
\string{aggregadgets = `295'} % \comment{54 pages}
\string{aggregadgets-first-page = `297'}
\string{gadgets = `347'} % \comment{116 pages}
\string{gadgets-first-page = `349'}
\string{debug = `461'} % \comment{20 pages}
\string{debug-first-page = `463'}
\string{demos = `481'} % \comment{10 pages}
\string{demos-first-page = `483'}
\string{sampleprog = `491'} % \comment{14 pages}
\string{sampleprog-first-page = `493'}
\string{gilt = `505'} % \comment{20 pages}
\string{gilt-first-page = `507'}
\string{c32 = `525'} % \comment{12 pages}
\string{c32-first-page = `527'}
\string{lapidary = `537'} % \comment{36 pages}
\string{lapidary-first-page = `539'}
\string{hints = `573'} % \comment{8 pages}
\string{hints-first-page = `575'}
\string{GlobalIndex = `580'}

\set{page = c32-first-page}


\chapter{Overview of C32}
\label{c32}

\index{Spreadsheet in C32}
C32 \cite{C32} is an object and constraint editor for Garnet objects.  It
allows Garnet objects to be viewed and edited using a spreadsheet-like
interface.  Each object is viewed in a panel, where each row corresponds
to a slot in the object.  The value of a slot can be changed directly, by
editing the value shown in the corresponding row.  Constraints can be
edited textually in separate formula-editing windows.

C32 can be used as a stand-alone tool to create and edit Garnet objects.
It is possible, for example, to edit an existing object by typing its name
in the title of a C32 panel, or by pointing and clicking on an object with
the mouse.  In addition, C32 is integrated with Lapidary, which uses it
for several editing tasks that used to be handled specially.

Because it uses the spreadsheet paradigm, C32 is highly structured.  Each
object is represented as a panel, and all panels are organized
horizontally in one long window.  A horizontal scroll bar allows different
panels to be displayed in the window.  Panels that contain more than 12
slots are displayed with a vertical scroll bar, so more slots can be made
visible as desired.

C32 keeps the display of each panel up to date.  When a slot is modified
using C32, the values displayed for other dependent slots are modified
accordingly.  Even if objects are changed from outside C32 (using the Lisp
listener or the mouse, for example), their corresponding panels are always
kept up to date.



\chapter{Loading C32}

To load C32, you load the file `garnet-c32-loader' and then type
\begin{programexample}
(c32:do-go)
\end{programexample}
Note that if you are using Lapidary, you do not need to load C32 explicitly;
Lapidary does it automatically.

The function \pr{(c32:do-go)} creates two windows: the C32 Commands
window, which contains the main commands used to control C32, and the
spreadsheet window.  Initially, the spreadsheet window contains a single,
empty panel whose title reads `Object name:'.  The title of this panel can
be edited to the name of an object, which is then displayed in the panel.

The full syntax for \pr{do-go} is as follows:
\begin{programexample}
c32:do-go {\it \&key (startup-objects NIL) (test-p NIL) (start-event-loop-p T)}  & [{\it Function}]
\end{programexample}
If {\tt\char`\<}startup-objects{\tt\char`\>} is specified, it should be a list of Garnet objects.
When C32 is started, it creates a panel for each object in the list, plus
the empty panel.  If {\tt\char`\<}test-p{\tt\char`\>} is specified, a little test window is
created with a few objects in it.  C32 can then be used to edit the
objects in the window.  Setting {\tt\char`\<}start-event-loop-p{\tt\char`\>} to NIL cause C32 to
start up with the main-event loop not running.



\chapter{The Spreadsheet Window}

\index{C32 panels}
The spreadsheet window contains a list of panels, each displaying a Garnet
object.  Figure \ref{c32-spreadsheet} shows the spreadsheet window with a
panel for a \pr{label-text} object and the empty panel.  Each panel
consists of a vertical scroller (using for displaying more slots), a
title, and up to 12 rows.  Each row displays the contents of a slot.

\begin{figure}
\begin{center}
\graphic{Postscript=`c32/c32-spreadsheet.ps',magnify=.8,BoundingBox=File}\end{center}
\caption{C32 Spreadsheet Window with two panels.}
\tag{c32-spreadsheet}
\end{figure}

The title of each panel shows the name of the object displayed in the
panel.  The title can be edited by clicking the left mouse button over it,
and then using the normal Garnet text editing commands.  Type Return to go
ahead, and {\tt\char`\^}G if you do not want to make the change.  Entering the name of
a different object in a panel's title causes the panel to display the new
object.  If the object does not exist, C32 will ask you if you wish to
actually create a new object.  Setting the title of a panel to be empty
causes the panel to be removed from the spreadsheet window; the object
being displayed is unaffected, however.  Setting the title of the empty
panel to an object's name causes a new panel to be created for the object.
\index{creating objects}

The left half of each row displays the name of the slot.  The names of
local slots are shown in a roman face; the names of inherited slots are
shown in italics.  Slots that contain a formula are indicated by an `F' on
a dark circle.  If the formula was inherited, this is indicated by an
`I' inside a circle, next to the formula symbol.  The right half of each
row displays the value of the slot.  If the value is inherited, an `I'
inside a circle is shown at the far right of the row.

At any time, you can have a primary selection and a secondary selection.
The primary selection is shown by a dark background, and is used for the
majority of C32 operations that require a slot or an object.  You may
change the primary selection by clicking the left mouse button over a slot
name, i.e., in the left side of a row.  The secondary selection is shown
by a gray outline around a slot, and is used for operations that require
two slots.  You may change the secondary selection by clicking the middle
mouse button over a slot name.  Both primary and secondary selections are
toggles, and can be eliminated by clicking over them.

Panels allow you to modify objects, as well as displaying them.  The value
of a slot can be edited by editing its text: first click on the value (in
the right half of the row) to get a text cursor, and then use the normal
Garnet text editing commands.  When you type Return, the slot is set to
the new value and the object is modified (type {\tt\char`\^}G if you do not want to
make the change).  Note that the package shown in the Commands Window is
used when reading the value you type in.  Setting the package
appropriately ensures that you do not have to type package qualifiers for
every function and symbol.

The last row of a panel is always empty.  Clicking in its left-hand half
allows you to add a slot to the object, or to display a slot that is
currently not shown.  When you click, you start editing an (initially
empty) slot name.  If the slot is currently not shown, its current value
is displayed.  If the slot is not present, it is created.  Its initial
value is inherited, if possible; otherwise, it is set to NIL.  You may
then edit the value (in the right-hand side).  As a special shortcut, it
is also possible to enter a slot name and a value together; just type the
slot name, a space, and then the value.

The formula associated with a slot can also be edited using the
spreadsheet window.  Click on the `F' symbol; this pops up a window that
allows the formula to be edited, as explained below.  This mechanism also
let you create formulas for slots that do not yet contain one: simply
click on the place where the `F' symbol would be, and a new formula window
will appear.  You may then type the text for the new formula.  Note that
editing the value of a slot that contains a formula is equivalent to doing
an \pr{s-value} on the slot with the new value: the old value is
temporarily replaced, but the formula is unaffected.


\chapter{Editing Formulas}

\index{editing formulas in C32}
Formulas can be edited in special editing windows.  When you click on the
`F' symbol of a slot in the spreadsheet window, a window is created in
which you can edit the textual representation of the formula.  If you
click on the `F' symbol and a window already displays that formula, the
window is simply moved to the front.  If you click on the `F' symbol of a
slot that contains a value, the value is shown as the initial text for the
(yet to be created) formula.

Figure \ref{c32-formula} shows the C32 formula window for the
\pr{:left} slot of the object shown in Figure \ref{c32-spreadsheet}.
A formula window contains a header, a vertical scroller, and a text
window.  The header displays the name of the object and the slot upon
which the formula is installed, and contains five buttons.  The
scroller allows you to examine different portions of long formulas.

\begin{figure}
\begin{center}
\graphic{Postscript=`c32/c32-formula.ps',magnify=.75,BoundingBox=File}\end{center}
\caption{The C32 formula window for the :left slot.}
\tag{c32-formula}
\end{figure}



When the formula window is created, the text cursor is initially
positioned at the top left of the text.  The cursor can be moved, and the
text can be edited, using the normal Garnet text editing operations.  Note
that typing the abort character ({\tt\char`\^}G) in a formula window has no effect;
click the Cancel button if you really want to abort the current changes to
the formula's text.

The five buttons in the header include the OK and Cancel button, plus
three buttons that can be used to reduce typing when the formula is being
edited.  The `OK' button installs the expression currently displayed in the
formula window into the slot of the object, and hides the formula window.
If errors are detected (for example, because the syntax in the formula
expression is illegal), you will see an error message and the formula
window will remain on the screen.  The `Cancel' button removes the window
without modifying the formula that is currently installed on the slot.

The `Insert Function' button pops up a menu with the names of functions
that are commonly used in formulas.  In addition to \pr{floor}, \pr{max},
and the like, the menu includes the functions from the \pr{opal:gv-}
family.  Select a function from the menu by clicking the left mouse button
over it; this will show the function's name in reverse video.  Clicking on
the `Insert Function' button will now insert the selected function,
enclosed in parentheses, at the cursor position in the formula window.

The `Insert From Spread' button inserts a reference to the object and slot
that are currently selected in the spreadsheet window into the formula
being edited.  C32 detects whether the selected object is the same as the
one that contains the formula, and if so, it generates a simple reference
using \pr{gvl}.

The `Insert From Mouse' button is similar, except that it allows you to
select the target object using the control-left mouse button.  The button
pops up a dialog box through which you may select an object.  C32 will
attempt to guess a slot; for example, if you click \pr{control-left} on
the left part of a string, it will insert the \pr{:left} slot.  If C32
cannot guess any slot, it leaves the slot name blank.  When the
appropriate object (and possibly slot) is shown in the dialog box,
clicking the `Apply' button will insert a reference into the formula
window.  Clicking the `OK' button will do the same, and hide the dialog
box as well.



\chapter{The Commands Window}

This window contains a menu with C32 commands that apply to the
spreadsheet window, and is shown in Figure \ref{c32-commands}.  Many of
the buttons in the Commands Window operate on the slot that is currently
selected in the spreadsheet window.  The window also displays the current
package that is used by C32 to interpret Lisp values and expressions (for
example, when you type a value or a formula).  The package can be changed
by editing the string in the box.

\begin{figure}
\begin{center}
\graphic{Postscript=`c32/c32-commands.ps',magnify=.75,BoundingBox=File}\end{center}
\caption{The C32 Commands Window.}
\tag{c32-commands}
\end{figure}

The meaning of each group of buttons is explained below.

\section{Point To Object}

This button pops up a dialog box that allows an object to be added to the
spreadsheet window by pointing and clicking with the mouse.  The dialog
box explains how to select objects (by clicking control-left) and how to
move from one object to another underneath it.  Clicking control-left on
any Garnet object inserts the name of the object in the dialog box.

Once the name of the desired object is displayed in the dialog box, you
can click the Apply button (which creates a new panel displaying the
object), the OK button (which does the same thing and then hides the
dialog box), or the Cancel button.

\section{Showing references to other slots}

Three buttons are used to show references, i.e., dependencies among slots.
Clicking on the `Slots Using Me' button displays green arrows that
indicate what formulas (in objects currently shown in C32) use the
selected slot.  The arrows originate on the Formula symbol of the
dependent slots, and point to the value portion of the selected slot.  If
a dependent slot belongs to another object, and that object is shown in a
panel, the arrow is drawn to the other panel.

Clicking on the `Slots I Use' button of a slot that contains a formula
shows red arrows from the formula symbol to all the slots (currently shown
in C32) upon which the formula depends.  These arrows are heavier than the
ones discussed previously, and they point to the slot side, rather than
the value side.

Clicking on the `Clear References' button eliminates all currently
displayed reference arrows.  This operation does not alter any internal
value, of course.


\section{Deleting, hiding, and showing slots}

The `Hide Slot' button causes the selected slot to be eliminated from the
C32 panel.  The value of the slot in the object is unaffected.  The `Show
All Slots' button causes all slots in the selected object to be displayed
in the panel.

The `Delete Slot' is used to delete the currently selected slot from the
actual object.  Care should be taken, because this is a destructive
operation.  Trying to delete some of the most important slots prompts for
confirmation.


\section{Copy Formula}

% \begin{comment}
% \index{generalizing formulas}
% The `Generalize Formula' button allows a function to be created by
% generalizing the formula associated with the current slot.  Generalizing a
% formula means that a function is created in which hard-wired references to
% objects and slots are replaced by parameters.  The function can then be
% used as a more general expression, for example inside other formulas.
% This button pops up a dialog box that allows you to type the name of the
% function to be created, and to select names for object and slot
% references.  When you click OK, the definition of the new function is
% printed in the Lisp listener window.
% \end{comment}
\index{copying formulas in C32}
The `Copy Formula' button copies the formula installed on the
secondary-selected slot to the slot that corresponds to the primary
selection.  A box pops up to make sure this is what you want to do.


\section{Quit}

This button causes C32 to destroy all its windows and exit.  If C32 was
started up from Lapidary, however, the windows are simply temporarily
hidden, so that C32 can start up faster the next time.



\chapter{C32 Internals}

\index{slots-to-show slot}
The list of slots to be displayed in a panel is kept in the
\pr{:slots-to-show} slot of Garnet objects.  In most cases, this slot is
inherited.  If you are creating new types of objects, you may want to set
the \pr{:slots-to-show} slot appropriately in the prototype, so that C32
will display only relevant slots when it shows an instance in a panel.

It is probably a good idea not to try to edit the contents of the
\pr{:slots-to-show} slot using C32 itself.

The current version of C32 is not very optimized; redisplaying and
scrolling panels, in particular, is rather inefficient.  We hope to make
its performance better in the next release.


\chapter*{References}
\bibliography
