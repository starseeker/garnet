\device{postscript}
\make{manual}
\disable{figurecontents}
\libraryfile{Garnet}
\libraryfile{Stable}
\style{spacing 1.0}
\string{TitleString = `Garnet Applications'}
\modify{Hd1, PageBreak Off}
\modify{description, size 9}
% \comment{****}
\begin{titlepage}
\begin{titlebox}
\vspace{0.6 inch}
\bg{Pictures of Applications
Using Garnet}

{\bf Edited by
Brad A. Myers}
\vspace{0.3 line}
\value{date}
\end{titlebox}
\vspace{0.5 inch}
\begin{center}
{\bf Abstract}\end{center}
\begin{text}
This is a collection of pictures of real applications written by Garnet users.
You can print just this section of the manual on a color printer to see some
dramatic color images.

\vspace{0.5 inch}
\begin{center}
Copyright \copyright{1994} - Carnegie Mellon University
\end{center}

%\vspace{2 lines}
The original Garnet research
was sponsored by NCCOSC under Contract No.
N66001-94-C-6037, Arpa Order No. B326.
The views and conclusions contained in this document are those of
the authors and should not be interpreted as representing the
official policies, either expressed or implied, of NCCOSC or the U.S.
Government.

Modifications Copyright \copyright{2005} by Robert P. Goldman and other
contributors.  All modifications to this document are made freely
available under the terms of the Lesser Gnu Public License.


\end{text}
\end{titlepage}


\string{overview = `1'} % \comment{26 pages}
\string{overview-first-page = `3'}
\string{apps = `27'} % \comment{12 pages}
\string{apps-first-page = `29'}
\string{tour = `41'} % \comment{20 pages}
\string{tour-first-page = `43'}
\string{tutorial = `61'} % \comment{42 pages}
\string{tutorial-first-page = `63'}
\string{kr = `101'} % \comment{52 pages}
\string{kr-first-page = `103'}
\string{opal = `151'} % \comment{70 pages}
\string{opal-first-page = `153'}
\string{inter = `219'} % \comment{78 pages}
\string{inter-first-page = `221'}
\string{aggregadgets = `295'} % \comment{54 pages}
\string{aggregadgets-first-page = `297'}
\string{gadgets = `347'} % \comment{116 pages}
\string{gadgets-first-page = `349'}
\string{debug = `461'} % \comment{20 pages}
\string{debug-first-page = `463'}
\string{demos = `481'} % \comment{10 pages}
\string{demos-first-page = `483'}
\string{sampleprog = `491'} % \comment{14 pages}
\string{sampleprog-first-page = `493'}
\string{gilt = `505'} % \comment{20 pages}
\string{gilt-first-page = `507'}
\string{c32 = `525'} % \comment{12 pages}
\string{c32-first-page = `527'}
\string{lapidary = `537'} % \comment{36 pages}
\string{lapidary-first-page = `539'}
\string{hints = `573'} % \comment{8 pages}
\string{hints-first-page = `575'}
\string{GlobalIndex = `580'}

\set{Page = apps-first-page}


% \comment{ Use \MajorHeading instead of \Chapter to avoid Table of Contents! }
\majorheading{Some Garnet Applications}

This section presents some examples of applications written using
Garnet.  As of September 29, 1993, we know of over 50 applications
written using Garnet, listed below.  If you
are using Garnet and you are not in the list, please send mail to
\pr{garnetjcs.cmu.edu}.  The figures on the following pages show some
of the applications.  These were generated using the Garnet utility
\pr{Opal:make-ps-file}.  Many of these figures are in color, so if you
have a color printer or postscript previewer, you might try looking at
this file in color.
\define{figcol, columns=3, boxed, justification=off,
        flushleft, size=7, spacing = 7 points}
\begin{figure}
\begin{figcol}
{\bf Companies}
1. Corporation For Open Systems
      Automated Protocol Analysis/Reference Tool
      Frank J. Wroblewski
2. Design Research Institute (Cornell \& Xerox)
      (various)
      Jim Davis
3. Deutches Forschungszentrum fuer Kuenstliche
          Intelligenz GmbH
      COSMA
      Stephen P. Spackman
4. GE Research and Development Center
      Metallurgical Expert Systems for Manufacturing
      K. J. Meltsner
5. GE Research and Development Center
      SKETCHER
      K. J. Meltsner
6. Hughes AI Center
      NLP Project
      Seth Goldman or Charles Dolan
7. Institute of Knowledge Engineering, Madrid, Spain
      KIISS: Knowledge Interface Interactive
          Surgery System
      Roberto Moriyon
8. Lawrence Livermore National Lab
      PLANET (Pump Layout ANd Evaluation Tool)
      Tom Canales
9. Microelectronics \& Computer Tech. Corp. (MCC)
      Scan: Intelligent Text Retrieval
      Elaine Rich
10. MITRE Corporation
      AIMI (An Intelligent Multimedia Interface)
      John D. Burger
11. National Science Center Foundation
      Learning Logic
      Lawrence Freil
12. PTT Research, Groningen, The Netherlands
      R\&D Employee Information Retrieval Service
      Colin Tattersall
13. Rank Xerox EuroPARC
      (various CSCW-related projects)
      Paul Dourish
14. StatSci
      GRAPHICAL-BELIEF
      Russell Almond
15. Transarc Corporation
      Lisp GUI for Encina
      Mark Sherman
16. U S WEST Advanced Technologies
      KBNL natural language system
      Randall Sparks
17. USC/ISI
      Humanoid
      Pedro Szekely
18. USC/ISI
      SHELTER knowledge-based development environ.
      Pedro Szekely

\newcolumn{}
{\bf Universities}
19. Carnegie Mellon University, CS
      Miro
      J. D. Tygar and J. M. Wing
20. Carnegie Mellon University, CS
      Learning Calendar System
      Conrad Poelman
21. Carnegie Mellon University, CS
      Interactive Fiction Editor
      Merrick Furst
22. Carnegie Mellon University, CS
      Redstone
      Jeff Schlimmer
23. Carnegie Mellon University, CS
      PURSUIT
      Francesmary Modugno
24. Carnegie Mellon University, ERDC
      LOOS
      Ulrich Flemming or Robert Coyne
25. Carnegie Mellon University, Math
      Educational Theorem Proving System
      Peter Andrews
26. Carnegie Mellon University, Psychology
      Soar Graphics Interface
      Frank Ritter
27. Carnegie Mellon University, Robotics
      MICRO-BOSS
      Norman Sadeh
28. Carnegie Mellon University, Robotics
      SAGE Data Visualization
      Steve Roth
29. Georgia Institute of Technology
      KritikTutor (interactive learning environment)
      Andres Gomez de Silva Garza, Nathalie Grue
30. MIT, Dept. of Brain and Cognitive Sciences
      SURF-HIPPO Neuron Simulator
      Lyle J. Borg-Graham
31. MIT, LCS, Computation Structures Group
      Debugging tools for the Id Language
      Steve Glim and R. Paul Johnson
32. New York University, Courant Institute
      COMLEX (Common Lexicon) dictionary project
      R. Grishman, C. Macleod, S. Wolff, and K. Rajan
33. Oregon State University
      Forms/3 (visual language)
      Margaret Burnett
34. Oregon State University, Dept. of CS
      CarGo (Game of Go)
      Peter Dudey
35. State University of New York at Buffalo, CS
      Air Battle Simulation
      Henry Hexmoor
36. State University of New York at Buffalo, CS
      SNePS Graphical UI
      John S. Lewocz
37. Tulane University, CS
      Natural Language Processing
      Robert Goldman
38. Tulane University, CS
      THESEUS
      Raymond Lang
\newcolumn{}
39. Universidad de las Americas-Puebla (Mexico)
      Visual Language
      Miguel Sanchez
40. University College London
      The Cognitive Browser
      Gordon Joly
41. University of Calif. at Santa Barbara, CS \& ECE
      Graphical Tools for the Development of
           Concurrent Systems
      L. Dillon, G. Kutty, P. Melliar-Smith,
           L. Moser, Y. Ramakrishna
42. University of Chicago, CS
      Shopper (planning and visual navigation)
      Daniel Fu
43. University of Chicago, CS
      SEAL (situation-driven execution and
           constructivist learning)
      Charles Earl
44. University of Chicago, AI Lab
      Roentgen (case-based aid for radiation therapy)
      Jeff Berger
45. University of Leeds
      Graphical Multi-User Domain Designer
      Roderick J. Williams
46. University of Leeds
      CLARE
      Nikos Drakos
47. University of Leeds
      ADVISOR
      Andrew J. Cole
48. University of Leeds
      PORSCHE
      Colin Tattersall
49. University of Michigan, EE\&CS
      Science works: Nuclear Engineering
      K. Abotel, E. Solloway, C. Quintaina, W. Martin
50. University of Queensland, Australia
      Conceptual Modeling CASE tool
      Anthony Berglas
51. University of Queensland, Australia
      Interoperating Network Servers
      Hung Wing
52. University of Saskatchewan
      DISCUS (DIstributed Computing at the U of S)
      Beth Protsko (User Interface only)
53. University of Southern California
      Dynamic Aggregation in Qualitative Simulation
      Nicolas Rouquette
54. University of Southern California
      Revenge: Knowledge-based S/W for
           (Re)Design-for-Assembly
      Gerard Kim,  Dr. George Bekey,
55. University of Washington, CS
      Multi-Garnet
      Michael Sannella
56. University of Washington, CS
      Electronic Encyclopedia Exploratorium
      Mike Salisbury
\end{figcol}
\end{figure}

% \comment{---------------  drakosclare ----------------------}
\newpage{}
\begin{center}
\graphic{Postscript=`apps/drakosnewclare.ps', magnify=0.6,
         boundingbox=file}\end{center}
\begin{center}
\graphic{Postscript=`apps/drakosclaregraph.ps', magnify=0.5,
         boundingbox=file}\end{center}
\begin{tabular}
{\bf Nikos Drakos}\\
Computer Based Learning Unit, University of Leeds, UK.\\
{\it CLARE}\\
\pr{nikosjcbl.leeds.ac.uk}\\
\end{tabular}

A collaborative project on an environment for the specification, testing,
maintenance and automatic generation of application software.
The context is batch process control in Chemical Engineering although
it is envisaged that the applicability of the environment will be more general.
A `domain expert' will be able to specify knowledge about plant
subsystems, plant configurations, and the allowable generic operations and
constraints on each plant subsystem. An `application engineer' will then use
the system to `glue' together predefined operations in order to make specific
products. The system will then generate process control code for particular
target hardware. Garnet is being used to capture and visualise plant and
process information through schematics, process diagrams, interactive
simulations and simple animations.
\begin{description}
\item[] Nikos Drakos, ``Object Orientation and Visual Programming'', in
Mamdouh Ibrahim, editor, {\it OOPSLA '92 Workshop on Object-Oriented
Programming Languages: The Next Generation}, Vancouver, B.C.
Canada, October 18 1992. Extended Abstract. pp. 85-93.
\end{description}



% \comment{---------------  ge mesh ----------------------}
\newpage{}

\begin{center}
\graphic{Postscript=`apps/ge.ps', magnify=1.0, boundingbox=file}\end{center}
\begin{tabular}
{\bf Kenneth Meltsner}\\
General Electric Company, Corporate Research and Development\\
{\it Metallurgical Expert System}\\
\pr{meltsnerjcrd.ge.com}\\
\end{tabular}
A mesh created using a virtual aggregate for the
polygons and another virtual aggregate for the square knobs.  For the
polygons, the virtual aggregate is passed a prototype for a
polygon, and an array containing the list of points and the color for
each polygon.   The virtual aggregate then pretends to
allocate an object for each element of the array, but actually just
draws the prototype object repeatedly.

\begin{description}
\item[] Kenneth J. Meltsner.
``A Metallurgical Expert System for Interpreting FEA,'' {\it Journal of Metals},
Oct, 1991, vol. 43, no. 10, pp. 15-17.
\end{description}

\newpage{}

% \comment{--------------- Lang THESEUS  ----------------------}
\begin{center}
\graphic{Postscript=`apps/lang1.ps', magnify=0.55, boundingbox=file}\end{center}
\begin{center}
\graphic{Postscript=`apps/lang2.ps', magnify=0.4, boundingbox=file}\end{center}
\begin{center}
\graphic{Postscript=`apps/lang3.ps', magnify=0.6, boundingbox=file}\end{center}
\begin{tabular}
{\bf Raymond Lang}\\
Tulane University, Computer Science Department\\
{\it THESEUS}\\
\pr{langjrex.cs.tulane.edu}\\
\end{tabular}

These are images of windows from the
THESEUS application used by the Tulane University Computer Science
Department on guided tours of the department given to visiting high
school seniors and other interested parties.  THESEUS is intended to
be used as part of a presentation on what the study of computer
science entails.  It does this by showing graphically the progress and
results of common search methods applied to the problem of finding the
exit of a randomly created maze.  THESEUS was developed in CMU Common
Lisp version 16d and the Garnet X-Windows toolkit version 2.01.

\begin{description}
R. Raymond Lang, {\it THESEUS: Using Maze Search to
\item[] Introduce Computer Science}.  Technical Report.  Computer Science
Department, Tulane University. 1992.
\end{description}

\newpage{}

% \comment{--------------- Berger Roentgen  ----------------------}
\begin{center}
\graphic{Postscript=`apps/bergerdose.ps', magnify=1, boundingbox=file}\end{center}
\begin{tabular}
{\bf Jeff Berger}\\
University of Chicago, Artificial Intelligence Laboratory\\
{\it Roentgen}\\
\pr{bergerjcs.uchicago.edu}\\
\end{tabular}
Roentgen is a case-based aid to radiation therapy planning.  It relies on
an archive of past therapy cases to suggest plans for new therapy
patients. Roentgen supports therapy planning by: 1) retrieving the case
which best matches the geometry and treatment constraints of the new
patient; 2) tailoring the plan to the specific details of the patient;
3) evaluating the results of applying the plan; 4) repairing the plan
to avoid any discovered faults in treatment results. This final plan
is the system's suggestion to the human planner. Roentgen breaks new
ground in solving problems in a domain dominated by spatial reasoning
and the satisficing of constraints.


\newpage{}

% \comment{--------------- lyleSurfHippo.ps  ----------------------}
\begin{center}
\graphic{Postscript=`apps/lyleSurfHippo.ps', magnify=0.7, boundingbox=file}\end{center}
\begin{tabular}
{\bf Lyle J. Borg-Graham}\\
MIT Dept. of Brain and Cognitive Sciences\\
{\it Surf-Hippo}\\
\pr{lylejai.mit.edu}\\
\end{tabular}
The SURF-HIPPO Neuron Simulator is a circuit  simulation  package  for
investigating morphometrically and biophysically detailed models of single
neurons and small networks of neurons. SURF-HIPPO allows ready construction of
multiple cells from various file formats, which can describe complicated
dendritic trees in 3-space with distributed non-linearities and synaptic
contacts between cells. Cell geometries may also be traced from the histology
directly on the screen, using the mouse.   An extensive user interface is
provided, including menus, 3D graphics of dendritic trees, and data plotting.
Data files may also be saved for analysis with external tools.  A
research version of SURF-HIPPO
(available by anonymous ftp from ftp.ai.mit.edu [pub/surf-hippo]) is written in
LISP, and is configured to run using the public domain CMU Common Lisp and
Garnet packages. Our version is compiled for SPARC workstations, and should be
easily ported to other UNIX machines running X. LISP is a useful simulator
language because it has the benefits of a powerful interpreted script language,
but it may also be compiled. Thus it is convenient to integrate custom code
into SURF-HIPPO. The simulator may also be used with a minimum of programming
expertise, if desired.
\begin{description}
\item[] Borg-Graham, L. and Grzywacz, N. M. ``A Model of the Direction
Selectivity Circuit in Retina: Transformations by Neurons Singly and
in Concert,'' in {\it Single Neuron Computation}, edited by T.
McKenna, J. Davis, and S. F. Zornetzer. Academic Press, 1992.
\end{description}

\newpage{}

% \comment{--------------- Graphical MUD rjwilliams1.ps ----------------------}
\begin{center}
\graphic{Postscript=`apps/rjwilliams1.ps', magnify=0.9, boundingbox=file}\end{center}
\begin{tabular}
{\bf Roderick J. Williams}\\
The University of Leeds, Leeds, LS2 9JT, UK.\\
{\it GMD (Graphical Mud (Multi-User Domain) Designer)}\\
\pr{rodwjcbl.leeds.ac.uk}\\
\end{tabular}
This application is aimed at supporting the creation of text-based multi-user
domains. Current techniques use text-based tools to create these environments,
but these tools have very little computer support so complexity and consistency
are sacrificed.
Our new application supports the graphical creation of MUD areas and enforces
topological constraints together with hierarchical grouping of features.
The graphical tool can be used in a number of modes which allow the information
to be filtered, zoomed and viewed in 2.5 D. Areas created can be printed and
additionally they can be saved as native code that can be executed.

\newpage{}



% \comment{--------------- sage.ps  ----------------------}
% \comment{l, bot, l->r,b->t}
\begin{center}
\graphic{Postscript=`apps/sage1.ps',
        rotate 90, magnify=0.375,
        boundingbox=`1.0in, 1.8125in, 7.4375in,
        9.25in'} \graphic{Postscript=`apps/sage2.ps', magnify=0.45,
        boundingbox=file}\end{center}
\begin{center}
\graphic{Postscript=`apps/sageNapolean.ps',
        rotate 90, magnify=0.6,
        boundingbox=`1.0in, 1.8125in, 7.4375in, 9.25in'}\end{center}
\begin{tabular}
{\bf Steven F. Roth}\\
Carnegie Mellon University, Robotics Institute\\
{\it SAGE}\\
\pr{rothjisl1.ri.cmu.edu}\\
\end{tabular}
The SAGE project is developing systems which automate the process of
designing presentations of information.  An automatic presentation system is
an intelligent interface component which receives information from a user or
application program and designs a combination of graphics and text that
effectively conveys it.  It's  purpose is to assume as much responsibility
for designing displays as required by a user, from layout and color
decisions to broader decisions about the types of charts, tables and
networks that can be composed within a display.  The SAGE project is
developing an interactive data exploration environment which contains
automatic display design capabilities integrated with data navigation,
manipulation and modification tools. These tools are being used to
explore large amounts of diverse data from marketing, logistical, real
estate, census and other databases.
\begin{description}
\item[] Roth, S.F. \& Mattis, J.A.  `Data Characterization for Intelligent
Graphics Presentation', In {\it CHI'90: Proceedings of the ACM/SIGCHI Conference
on Computer Human Interaction}, Seattle, April, 1990. pages 193-200.
\end{description}
\newpage{}


% \comment{--------------- salisburgencl.ps  ----------------------}
\begin{center}
\graphic{Postscript=`apps/salisburgencl.ps', magnify=0.9, boundingbox=file}\end{center}
\begin{tabular}
{\bf Mike Salisbury}\\
University of Washington, Department of Computer Science\\
{\it Electronic Encyclopedia Exploratorium}\\
\pr{salisburjcs.washington.edu}\\
\end{tabular}
The Electronic Encyclopedia Exploratorium is an electronic
how-things-work book.  It allows the user to learn about devices
by experimenting with the components of
those devices in a lab simulation setting.  A causal model simulator
lies beneath the user
interface which simulates the current device and can provide
causal explanations of the results of that simulation.  Other
high-level tools are planned for future enhancement.
\begin{description}
\item[] F. G. Amador, D. Berman, A. Borning, T. DeRose, A. Finkelstein, D.
Neville, Norge, D. Notkin, D. Salesin, M. Salisbury, J. Sherman, Y.
Sun, D. S. Weld, and G. Winkenbach.  {\it Electronic `How Things Work'
Articles: A Preliminary Report}.  University of
Washington, Department of Computer Science and Engineering Technical
Report 92-04-08. June, 1992.
\end{description}

\newpage{}



% \comment{--------------- Soar  ----------------------}
\begin{center}
\graphic{Postscript=`apps/soar.ps', magnify=1.0, boundingbox=file}\end{center}
\begin{tabular}
{\bf Frank E. Ritter}\\
Department of Psychology, U. of Nottingham\\
{\it The Developmental Soar Interface}\\
\pr{Ritterjpsyc.nott.ac.uk}\\
\end{tabular}
The Developmental Soar Interface provides a graphical and
textual interface to observe and modify models (programs) for Soar, an
AI programming language that also realizes a unified theory of
cognition.  Garnet is used to graphically represent Soar's goal stack
and internal state, and to help users modify and observe structures in
Soar.
\begin{description}
Ritter, F. E. (1993) {\it TBPA: A methodology and software environment for
\item[] testing process models' sequential predictions with protocols}, PhD
thesis, Department of Psychology, Carnegie-Mellon University.
Reprinted as techreport CMU-CS-93-101, Carnegie-Mellon University.
\end{description}

\newpage{}


% \comment{--------------- spackmancosma.ps  ----------------------}
% \comment{l, bot, l->r,b->t}
\begin{center}
\graphic{Postscript=`apps/spackmancosma.ps',
        rotate 90, magnify=0.7,
        boundingbox=`1.6875in, 1.0625in, 6.75in, 10.0in'}\end{center}
\begin{tabular}
{\bf Stephen P. Spackman}\\
Projekt DISCO\\
Deutches Forschungszentrum fuer Kuenstliche Intelligenz GmbH\\
{\it COSMA, the CoOperative Scheduling Management Agent}\\
\pr{spackmanjdfki.uni-sb.de or stephenjacm.org}\\
\end{tabular}
The calendar window shows the dark bar of the past sweeping, one pixel
each half hour of the day and night, across a horizontal line [not
visible in this Postscript image] summarising by its width and height
the user's working hours and appointments, tentative and firm.  The
marginal time tags can be dragged up and down, and it will eventually be
possible to type over the top of them to jump to a given time.  The
datebook window presents an expanded view of time as an infinite tape
from which appointment forms can be popped up by pointing or by sweeping
out free areas.  Most importantly, when arrangements involve several
people the system communicates with its peers and with meeting
participants by reading and writing email in German; the displays are
updated in real time.

The fields of the appointment form are semi-structured: they can
be filled in with the help of menus - such as that visible on the lower
window - that drop down from the small icons on the right; numeric, date
and time values within them can be incremented and decremented directly
with mouse buttons; and experienced users can type structured values
straight in.  Unconstrained German text (the graphic interface will soon
be English/French/German trilingual, but the natural language parser and
generator speak only German) can also be entered. It is routed to the
natural language system for analysis; planned improvements to the
pragmatics module will allow you to give up on the structured form
completely and type informal questions and instructions into the notes
field, as you might for a human secretary who had stepped out of the
room.
\begin{tabular}
The work underlying this picture was supported by a research grant, FKZ ITW\\
9002 0, from the German Bundesministerium fuer Forschung und\\
Technologie to the DFKI project DISCO.\\
\end{tabular}
\begin{description}
\item[] Elizabeth A. Hinkelman and Stephen P. Spackman,
``Abductive Speech Act Recognition, Corporate Agents and the COSMA System,''
in {\it Abduction, Beliefs and Context: Proceedings of the second ESPRIT
PLUS workshop in computational pragmatics}.  W. J. Black and G. Sabah
and T. J. Wachtel, eds. Academic Press, 1992.
\end{description}


